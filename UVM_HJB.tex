% Preview source code

%% LyX 2.0.6 created this file.  For more info, see http://www.lyx.org/.
%% Do not edit unless you really know what you are doing.
\documentclass[english]{article}
\usepackage[T1]{fontenc}
\usepackage[latin9]{inputenc}
\usepackage{geometry}
\geometry{verbose,tmargin=2.5cm,bmargin=2.5cm,lmargin=2.5cm,rmargin=2.5cm,headheight=31.941cm}
\usepackage{babel}
\usepackage{float}
\usepackage{url}
\usepackage{amsmath}
\usepackage{amssymb}
\usepackage{graphicx}
\usepackage{esint}
\usepackage[unicode=true,
 bookmarks=true,bookmarksnumbered=false,bookmarksopen=false,
 breaklinks=false,pdfborder={0 0 1},backref=false,colorlinks=false]
 {hyperref}
\hypersetup{
 pdfauthor={Vathana.S.Leang}}

\makeatletter

%%%%%%%%%%%%%%%%%%%%%%%%%%%%%% LyX specific LaTeX commands.
%% Special footnote code from the package 'stblftnt.sty'
%% Author: Robin Fairbairns -- Last revised Dec 13 1996
\let\SF@@footnote\footnote
\def\footnote{\ifx\protect\@typeset@protect
    \expandafter\SF@@footnote
  \else
    \expandafter\SF@gobble@opt
  \fi
}
\expandafter\def\csname SF@gobble@opt \endcsname{\@ifnextchar[%]
  \SF@gobble@twobracket
  \@gobble
}
\edef\SF@gobble@opt{\noexpand\protect
  \expandafter\noexpand\csname SF@gobble@opt \endcsname}
\def\SF@gobble@twobracket[#1]#2{}

%%%%%%%%%%%%%%%%%%%%%%%%%%%%%% Textclass specific LaTeX commands.
\numberwithin{equation}{section}
\numberwithin{figure}{section}
\numberwithin{table}{section}

%%%%%%%%%%%%%%%%%%%%%%%%%%%%%% User specified LaTeX commands.
\usepackage[T1]{fontenc} % MiKTeX as ltxbase
\usepackage[latin9]{inputenc} % MiKTeX as ltxbase
\usepackage{lmodern} % for modern T1 fonts

% hyphenation patterns for English


% euro symbol
\usepackage{eurosym} % MiKTeX as eurosym

% used to prevent footnotes from disapearing when placed inside tables/other
\usepackage{footnote} % MiKTeX as mdwtools

% remove the default identation of new paragraphs
\usepackage{parskip} % MiKTeX as parskip


% for cross references
%\usepackage{hyperref} % MiKTeX as hyperref
\usepackage{bookmark} % MiKTeX as oberdiek

% for math
\usepackage{amsmath,amssymb,amsthm} % MiKTeX as amsmath, amscls
\usepackage{thmtools,thm-restate} % MiKTeX as thmtools

% for functions defined by branches
\usepackage{cases} % MiKTeX as cases

% for double stroke mathematical symbols
\usepackage{dsfont} % MiKTeX as dstroke

% to allow the use of appendices
\usepackage[titletoc,toc,title]{appendix} % MiKTeX as appendix

% to allow fixing a float in the place where it is declared
\usepackage{float} % MiKTeX as float

% used to color and highlight text. Used in the definition of \ins, \del, \hlt.
\usepackage[usenames,dvipsnames]{xcolor} % MiKTeX as xcolor

% insert comments in the PDF. Used in the definition of \cmt.
\usepackage{pdfcomment} % MiKTeX as pdfcomment, bezos

% decorate text. Used in the definition of \ins and \del.
\usepackage[normalem]{ulem} % MiKTeX as ulem

% makes figure and table captions prettier
\usepackage[font=small,labelfont=bf,margin=0.4cm]{caption} % MiKTeX as caption

% to include figures in the text
\usepackage{graphicx} % MiKTeX as graphics

% "Tikz is probably the most complex and powerful tool to create graphic elements."
% https://www.sharelatex.com/learn/TikZ_package
\usepackage{tikz} % MiKTeX as pgf

% packages used in the composition of the Scope of Approval part of the Valuation Model template
% multi-column typesetting mechanism
\usepackage{framed} % MiKTeX as framed
%\usepackage{mdframed} % MiKTeX as mdframed and l3kernel - the two are needed
%\def\changemargin#1#2{\list{}{\rightmargin#2\leftmargin#1}\item[]}
%\let\endchangemargin=\endlist
%\newmdenv[
  %skipabove=\parskip,
  %linewidth=0.8pt,
  %%innertopmargin = 0pt,
  %%leftmargin = 0pt,
  %%innerleftmargin = 1em,
  %%innerbottommargin = 0pt,
  %%innerrightmargin = 0pt,
  %%rightmargin = 0pt,
  %%topline = true,
  %%rightline = true,
  %%bottomline = true,
%]{simpleFrame}

% used in template development to fill in sections with random text
%\usepackage{lipsum} % MiKTeX as lipsum

% packages used to compose the header and footer of the MRG template
\usepackage{fancyhdr} % MiKTeX as fancyhdr

% to control itemized lists and enumerations
\usepackage{enumitem} %MiKTeX as enumitem

% standard array and tabular environments
\usepackage{array} % MiKTeX as tools

% packages to customize the tables of the MRG template
% colored table cells and headings
\usepackage{colortbl} % MiKTeX as colortbl
% tables that span over multiple pages
\usepackage{longtable} % MiKTeX as tools
% easier interface than tabular to control table size
\usepackage{tabularx} % MiKTeX as tools
% new column  definition to lighten the usage of tabularx
\newcolumntype{L}[1]{>{\hsize=#1\hsize\raggedright\arraybackslash}X}
\newcolumntype{R}[1]{>{\hsize=#1\hsize\raggedleft\arraybackslash}X}
\newcolumntype{C}[1]{>{\hsize=#1\hsize\centering\arraybackslash}X}
%\newcolumntype{C}{>{\centering\arraybackslash}X}
% writing professional tables
\usepackage{booktabs}

% The forloop package provides the command \forloop for the user.
\usepackage{forloop} % MiKTeX as forloop

\usepackage{xstring} % MiKTeX as xstring

\usepackage{etoolbox} % MiKTeX as etoolbox

\usepackage[ruled,vlined]{algorithm2e} % MiKTeX as algorithm2e

\usepackage{multirow} % MiKTeX as multirow

\makeatother

\begin{document}

\title{Uncertain Volatility in the HJM Framework}


\author{Vathana Leang}

\maketitle

\section{The HJM Framework%
\footnote{Demonstration in Appendix \ref{sub:DerivationHJM}%
}}

In the HJM framework, under the risk-neutral measure $\mathbb{Q}$,
the instantaneous forward rate is defined as follow. For $t\leq T$:
\[
f\left(t,T\right)=f_{0}\left(T\right)+k\left(t,T\right)\Phi_{t}\gamma\left(t,T\right)+k\left(t,T\right)\Gamma_{t}
\]


where 
\begin{itemize}
\item $T\rightarrow f_{0}\left(T\right)$ is the current instantaneous rate
curve bootstrapped from market instruments. This curve is assumed
to be deterministic. The first point $f_{0}\left(0\right)$ is today's
fixing.
\item $k\left(t,T\right)=\exp\left(-\int_{t}^{T}\beta\left(s\right)\mathrm{d}s\right)$
and $\gamma\left(t,T\right)=\int_{t}^{T}k\left(t,s\right)\mathrm{d}s$
with $T\rightarrow\beta\left(T\right)$ being the term structure of
the time-dependent mean-reversion parameter.
\item $\left(\Gamma,\Phi\right)_{t}$ is a Markovian system given by 
\begin{equation}
\begin{cases}
\mathrm{d}\Gamma_{t} & =\sigma_{t}\mathrm{d}W_{t}^{\mathbb{Q}}+\left(\Phi_{t}-\beta_{t}\Gamma_{t}\right)\mathrm{d}t\\
\mathrm{d}\Phi_{t} & =\left(\sigma_{t}^{2}-2\beta_{t}\Phi_{t}\right)\mathrm{d}t
\end{cases}\label{eq:Markov}
\end{equation}

\end{itemize}
The distribution of the instantaneous forward rate is fully characterized
by the mean-reversion $\beta$ and the instantaneous volatility $\sigma_{t}$
of the short rate.


\subsection{Zero-Coupon Bond\label{sub:Zero-Coupon-Bond}}

In th HJM framework, at time $t\leq T$, the price of zero-coupon
bond, maturing at time $T$ is given by: 
\begin{eqnarray}
B\left(t,T\right) & = & \exp\left[-\int_{t}^{T}f\left(t,s\right)\mathrm{d}s\right]\nonumber \\
 & = & \exp\left[-\int_{t}^{T}f_{0}\left(s\right)\mathrm{d}s\right]\exp\left[-\left(\frac{1}{2}\Phi\left(t\right)\gamma^{2}\left(t,T\right)+\Gamma\left(t\right)\gamma\left(t,T\right)\right)\right]\label{eq:bondPrice}
\end{eqnarray}


The price can also be computed as follow
\begin{equation}
\begin{cases}
u^{bond}\left(t,x,y\right) & =\mathbb{E}_{t}^{\mathbb{Q}}\left[\exp\left(-\int_{t}^{T}f\left(s,s\right)\mathrm{d}s\right)|\Gamma\left(t\right)=x,\Phi\left(t\right)=y\right]\\
u^{bond}\left(T,x,y\right) & =1
\end{cases}\label{eq:bond}
\end{equation}


where $u^{bond}$ is the solution of the PDE associated with the system
$\left(\ref{eq:Markov}\right)$ with the terminal condition defined
in $\left(\ref{eq:bond}\right)$.

Note that the bond price is independent from the volatility process.


\subsection{LIBOR and Vanilla Options Indexed on the LIBOR}

Using the same notations defined in the section \ref{sub:Zero-Coupon-Bond},
the forward rate seen at time $t\leq T$ over the period $\left[T,T+\delta\right]$
is defined by 
\[
L\left(t,T,T+\delta\right)=\frac{1}{\delta}\left[\frac{B\left(t,T\right)}{B\left(t,T+\delta\right)}-1\right]
\]


Considering a product paying at time $T+\delta$ a function $g$ of
the LIBOR with tenor $\delta$ fixing at time $T$. 

Its value at time $t$ is given by
\begin{equation}
\begin{cases}
u\left(t,x,y\right) & =\mathbb{\mathbb{E}}_{t}^{\mathbb{Q}}\left[\delta\exp\left(-\int_{t}^{T+\delta}f\left(s,s\right)\mathrm{d}s\right)g\left(L\left(T,T,T+\delta\right)\right)|\Gamma\left(t\right)=x,\Phi\left(t\right)=y\right]\\
u\left(T,x,y\right) & =\delta g\left(L\left(T,T,T+\delta\right)\right)B\left(T.T+\delta\right)
\end{cases}\label{eq:LIBORderivative}
\end{equation}


where $u$ is the solution of the PDE associated with the system $\left(\ref{eq:Markov}\right)$
with the terminal condition defined in $\left(\ref{eq:LIBORderivative}\right)$.

Note that the forward LIBOR (g is the identity) is independent from
the volatility process.


\subsection{Swap Rate and Vanilla Options Indexed on the Swap Rate}

Using the same notations defined in the section \ref{sub:Zero-Coupon-Bond},
the swap rate associated with a fixed for float swap is given by
\[
S_{t}=\frac{\sum_{j=1}^{M}\beta_{j}B\left(t,\tau_{j+1}\right)L\left(t,\tau_{j},\tau_{j+1}\right)}{\sum_{i=1}^{N}\alpha_{i}B\left(t,T_{i}\right)}
\]


where
\begin{itemize}
\item $\alpha_{i}$ and $\beta_{j}$ are the day count fractions of the
fixed and the float leg.
\item $\tau_{1}=T_{1}$ is the first payment date.
\item $\tau_{M}=T_{N}$ is the last payment date.
\end{itemize}
Considering a product paying multiple cash flow with a constant coupon
rate being a function $g$ of the swap rate fixing at the $T_{1}$.
The payment dates of the product coincide with the ones of the swap
rate.

Its value at time $t<T_{1}$ under the risk-neutral measure $\mathbb{Q}$
is given by
\begin{equation}
\begin{cases}
u\left(t,x,y\right) & =\mathbb{E}_{t}^{\mathbb{Q}}\left[\sum_{i=1}^{N}\alpha_{i}\exp\left(-\int_{t}^{T_{i}}f\left(s,s\right)\mathrm{d}s\right)g\left(S_{T}\right)|\Gamma\left(t\right)=x,\Phi\left(t\right)=y\right]\\
u\left(T,x,y\right) & =g\left(S_{T}\right)\sum_{i=1}^{N}\alpha_{i}B\left(t,T_{i}\right)
\end{cases}\label{eq:Swapderivative}
\end{equation}


where $u$ is the solution of the PDE associated with the system $\left(\ref{eq:Markov}\right)$
with the terminal condition defined in $\left(\ref{eq:Swapderivative}\right)$. 

Note that the forward swap rate is independent from the volatility
process. 


\subsection{Structured Products}

In this section we consider a product paying cash-flows with fixed
coupon rate or coupon rates being function of the LIBOR at successive
fixing dates $\left\{ \tau_{j}\right\} _{1\leq j\leq M}$. 

The value at time $t<\tau_{1}$ is given by: 
\[
u\left(t,x,y\right)=\mathbb{E}_{t}^{\mathbb{Q}}\left[\sum_{j=1}^{M}\beta_{j}\exp\left(-\int_{t}^{\tau_{j+1}}f\left(s,s\right)\mathrm{d}s\right)g\left(L\left(t,\tau_{j},\tau_{j+1}\right)\right)|\Gamma\left(t\right)=x,\Phi\left(t\right)=y\right]
\]


In this setting, by lineary, the expectation $\mathbb{E}$ and the
sum $\sum$ can be swapped and the different cash-flows can be computed
independently. 

In the Uncertain Volatility Framework introduced in the section \ref{sec:The-Uncertain-Volatility},
the symbol swap requires caution.


\section{The Uncertain Volatility Framework\label{sec:The-Uncertain-Volatility}}

As stated in the previous section, the instantaneous forward curve
as well as the pricing measure $\mathbb{Q}$ are dependent on the
choice of the pair $\left(\beta,\sigma_{t}\right)$.\\


From now on, the mean-reversion $\beta$ is supposed to be constant
and time-independent. A pricing measure $\mathbb{Q}^{\sigma_{t}}$
is then associated to a given volatility process $\sigma_{t}$. Let
us notice that one volatility process can be associated to more than
one pricing measure (if $\sigma_{t}$ is driven by its own Brownian
motion for example).\\


In the Uncertain Volatility Framework, rather than specifying the
dynamics of the volatility (e.g. considering a deterministic and time-independent
constant volatility), a family of volatility processes is considered
instead. For example, we can consider all the possible volatility
processes taking values in a given range $\left[\sigma_{inf},\sigma_{sup}\right]$.
The specification of the family is thus reduced to two parameters.
The methodology to choose those parameters will be discussed later.
Finally us note that the two parameters can be made time-dependent
to account for different volatility regime. \\


Formally:
\begin{itemize}
\item Let $\Sigma_{\left[\sigma_{inf},\sigma_{sup}\right]}^{\left[0,T\right]}$
be the set of all volatility processes $\sigma_{t}$ taking values
in the range $\left[\sigma_{inf},\sigma_{sup}\right]$ for each $t\in\left[0,T\right]$.
\item Let $\mathcal{Q}_{\left[\sigma_{inf},\sigma_{sup}\right]}^{\left[0,T\right]}$
be the set of all risk-neutral measures associated to all $\sigma_{t}\in\Sigma_{\left[\sigma_{inf},\sigma_{sup}\right]}^{\left[0,T\right]}$
\end{itemize}
Rather than computing one price associated to one pricing measure,
we will compute the upper and the lower bound prices for pricing measures
$\mathbb{Q}\in\mathcal{Q}_{\left[\sigma_{inf},\sigma_{sup}\right]}^{\left[0,T\right]}$.
This gives us a conservative range on the price of an asset given
the two bounds $\sigma_{inf}$ and $\sigma_{sup}$.


\subsection{Vanilla Instruments indexed on a single fixing of the LIBOR}

Considering a product paying at time $T+\delta$ a function $g$ of
the LIBOR with tenor $\delta$ fixing at time $T$. The upper and
lower bound prices are given by
\begin{equation}
\begin{cases}
u_{sup}\left(t,x,y\right) & =\sup_{\mathbb{Q}\in\mathcal{Q}_{\left[\sigma_{inf},\sigma_{sup}\right]}^{\left[0,T+\delta\right]}}\mathbb{E}_{t}^{\mathbb{Q}}\left[\exp\left(-\int_{t}^{T+\delta}f\left(s,s\right)\mathrm{d}s\right)g\left(L\left(T,T,T+\delta\right)\right)|\Gamma\left(t\right)=x,\Phi\left(t\right)=y\right]\\
u_{sup}\left(T,x,y\right) & =g\left(L\left(T,T,T+\delta\right)\right)B\left(T.T+\delta\right)
\end{cases}\label{eq:supPrice}
\end{equation}
\begin{equation}
\begin{cases}
u_{inf}\left(t,x,y\right) & =\inf_{\mathbb{Q}\in\mathcal{Q}_{\left[\sigma_{inf},\sigma_{sup}\right]}^{\left[0,T+\delta\right]}}\mathbb{E}_{t}^{\mathbb{Q}}\left[\exp\left(-\int_{t}^{T+\delta}f\left(s,s\right)\mathrm{d}s\right)g\left(L\left(T,T,T+\delta\right)\right)|\Gamma\left(t\right)=x,\Phi\left(t\right)=y\right]\\
u_{inf}\left(T+\delta,x,y\right) & =g\left(L\left(T,T,T+\delta\right)\right)B\left(T.T+\delta\right)
\end{cases}\label{eq:infPrice}
\end{equation}


$u_{sup}$ and $u_{inf}$ are solutions of non-linear PDEs presented
in the section \ref{sub:Non-Linear-Pricing-PDE}.


\subsection{Vanilla Instruments indexed on a basket of fixings of the LIBOR}

Considering a multiple cash-flow product contingent on different fixings
of the LIBOR, in the Uncertain Volatility framework, the different
cashflows cannot be valued independently. Indeed, given that $\sup\sum\leq\sum\sup$
and $\inf\sum\geq\sum\inf$, the range of price obtained will not
be optimal (i.e. the most narrow range containing the prices under
the pricing measures of $\mathcal{Q}_{\left[\sigma_{inf},\sigma_{sup}\right]}$)
.\\


To get the optimal range, the product must be valued iteratively in
a backward fashion as follow. Considering the payment dates $\left\{ \tau_{j}\right\} _{1\leq j\leq M}$.
\begin{enumerate}
\item $u\left(\tau_{M},x,y\right)=0$, $\beta_{M+1}=0$
\item For $j$ from $M$ to $1$

\begin{enumerate}
\item $u\left(\tau_{j},x,y\right)=\beta_{j}B\left(\tau_{j},\tau_{j+1}\right)g\left(L\left(\tau_{j},\tau_{j},\tau_{j+1}\right)\right)+u\left(\tau_{j}^{+},x,y\right)$
\item For $t\in\left[\tau_{j-1},\tau_{j}\right[$, solve backwardly for
$u\left(t,x,y\right)$ with the updated terminal condition defined
above and the pricing PDE described in the next section
\end{enumerate}
\end{enumerate}


Intuitively, there exists a volatility process $\sigma_{t}$ that
is achieving the maximum and the minimum of the structure product
price. 

By starting from the last cash flow that prevails on the interval
$\left[\tau_{M-1},\tau_{M}\right]$, we are looking for the optimal
volatility process between $\left[\tau_{M-1},\tau_{M}\right]$. 

Once this is done, the process will be ``frozen'' on $\left[\tau_{M-1},\tau_{M}\right]$
given that the prices of the remaining cash-flows do not depend on
that portion of the volatility process. 

We move to the next interval, corresponding to the next cash-flow
and look for the optimal volatility on that interval.

Hence, Pricing a complex cash-flow products with the Uncertain volatility
requires to first divide the time interval with the payment dates
of each leg. 


\subsection{Vanilla Instruments indexed on a fixing of the Swap rate}

Considering a product paying multiple cash flow with a constant coupon
rate being a function $g$ of the swap rate fixing at the $T_{1}$.
The payment dates of the product coincide with the ones of the swap
rate. The upper and lower bound prices are given by

\begin{equation}
\begin{cases}
u_{sup}\left(t,x,y\right) & =\sup_{\mathbb{Q}\in\mathcal{Q}_{\left[\sigma_{inf},\sigma_{sup}\right]}^{\left[0,T+\delta\right]}}\mathbb{E}_{t}^{\mathbb{Q}}\left[\sum_{i=1}^{N}\alpha_{i}\exp\left(-\int_{t}^{T_{i}}f\left(s,s\right)\mathrm{d}s\right)g\left(S_{T}\right)|\Gamma\left(t\right)=x,\Phi\left(t\right)=y\right]\\
u_{sup}\left(T,x,y\right) & =g\left(S_{T}\right)\sum_{i=1}^{N}\alpha_{i}B\left(t,T_{i}\right)
\end{cases}\label{eq:supPrice-1}
\end{equation}
\begin{equation}
\begin{cases}
u_{inf}\left(t,x,y\right) & =\inf_{\mathbb{Q}\in\mathcal{Q}_{\left[\sigma_{inf},\sigma_{sup}\right]}^{\left[0,T+\delta\right]}}\mathbb{E}_{t}^{\mathbb{Q}}\left[\sum_{i=1}^{N}\alpha_{i}\exp\left(-\int_{t}^{T_{i}}f\left(s,s\right)\mathrm{d}s\right)g\left(S_{T}\right)|\Gamma\left(t\right)=x,\Phi\left(t\right)=y\right]\\
u_{inf}\left(T+\delta,x,y\right) & =g\left(S_{T}\right)\sum_{i=1}^{N}\alpha_{i}B\left(t,T_{i}\right)
\end{cases}\label{eq:infPrice-1}
\end{equation}


$u_{sup}$ and $u_{inf}$ are solutions of non-linear PDEs presented
in the section \ref{sub:Non-Linear-Pricing-PDE}.


\subsection{Non-Linear Pricing PDE\label{sub:Non-Linear-Pricing-PDE}}

$u_{inf}$ and $u_{sup}$ respectively satisfies the following non-linear
PDEs%
\footnote{Demonstraction in the Appendix \ref{sub:DerivationHJB}.%
}
\begin{equation}
\partial_{t}u+\left(y-\beta x\right)\cdot\partial_{x}u-2\beta\cdot y\cdot\partial_{y}u+\frac{1}{2}\left(\partial_{xx}u+2\partial_{y}u\right)\cdot\Sigma_{inf}^{2}\left(\partial_{xx}u+2\partial_{y}u\right)-\left(f_{0}\left(t\right)+x\right)\cdot u=0\label{eq:infPDE}
\end{equation}


with 
\[
\Sigma_{inf}\left(x\right)=\begin{cases}
\sigma_{inf} & x\ge0\\
\sigma_{sup} & x<0
\end{cases}
\]


\begin{equation}
\partial_{t}u+\left(y-\beta x\right)\cdot\partial_{x}u-2\beta\cdot y\cdot\partial_{y}u+\frac{1}{2}\left(\partial_{xx}u+2\partial_{y}u\right)\cdot\Sigma_{sup}^{2}\left(\partial_{xx}u+2\partial_{y}u\right)-\left(f_{0}\left(t\right)+x\right)\cdot u=0\label{eq:supPDE}
\end{equation}


with 
\[
\Sigma_{sup}\left(x\right)=\begin{cases}
\sigma_{sup} & x\ge0\\
\sigma_{inf} & x<0
\end{cases}
\]


The terminal condition of the PDEs corresponds to the payoff of the
various products. 


\section{Numerical Schemes}

Two discretization schemes will be used to solve the PDEs $\left(\ref{eq:infPDE}\right)$
and $\left(\ref{eq:supPDE}\right)$: the explicit scheme and the ADI
scheme.

The grid $\left(t,x,y\right)$ is assumed to be ``isotropic'' in
the sense that the x-axis is independent from coordinate $\left(t,y\right)$.
Same along the y-axis and the t-axis.

The partition of each axis is assumed to be regular for now. For $\left(i,j,n\right)\in\left[0,\dots,I\right]\times\left[0,\dots J\right]\times\left[0,\dots,N\right]$
\begin{eqnarray*}
x_{i} & = & x_{0}+i\cdot\delta x\\
y_{j} & = & y_{0}+j\cdot\delta y\\
t^{n} & = & t^{0}+n\cdot\delta t
\end{eqnarray*}
term 

The price $u\left(t,x,y\right)$ is discretized as follow
\[
u\left(t^{n},x_{i},y_{j}\right)=u_{i,j}^{n}
\]


The time and the space derivatives are approximated with finite difference
schemes.


\paragraph{Explicit Scheme}

Solve backward from $N$ to $0$, for each $\left(i,j\right)$

\begin{eqnarray*}
u_{i,j}^{n} & = & \left(1+\alpha_{0}^{n+1}+\alpha_{x}^{n+1}+\alpha_{y}^{n+1}\right)u_{i,j}^{n+1}+\beta^{n+1}u_{i+1.j}^{n+1}+\gamma^{n+1}u_{i-1,j}^{n+1}+\zeta^{n+1}u_{i,j+1}^{n+1}+\eta^{n+1}u_{i,j-1}^{n+1}
\end{eqnarray*}



\paragraph{ADI Scheme}

Solve the PDE backwardly from $N$ to $0$ in two steps. 
\begin{enumerate}
\item The scheme is explicit in the y-axis and implicit in the x-axis. For
each ``column'' $j$, we have an equation of the form $Ax=b$ where
$A$ is a tridiagonal matrix. We solve it to find the intermediate
$u^{n+\frac{1}{2}}$ price.
\[
\left(1+\frac{1}{2}\alpha_{0}^{n+1}+\frac{1}{2}\alpha_{y}^{n+1}\right)u_{i,j}^{n+1}+\frac{1}{2}\zeta^{n+1}u_{i,j+1}^{n+1}+\frac{1}{2}\eta^{n+1}u_{i,j-1}^{n+1}=\left(1-\frac{1}{2}\alpha_{x}^{n+\frac{1}{2}}\right)u_{i,j}^{n+\frac{1}{2}}-\frac{1}{2}\beta^{n+\frac{1}{2}}u_{i+1.j}^{n+\frac{1}{2}}-\frac{1}{2}\gamma^{n+\frac{1}{2}}u_{i-1,j}^{n+\frac{1}{2}}
\]

\item The scheme is implicit in the y-axis and explicit in the y-axis. For
each ``row'' $i$, we have an equation of the form $Ax=b$ where
$A$ is a tridiagonal matrix. We solve it to find the intermediate
$u^{n}$ price.
\end{enumerate}
\[
\left(1+\frac{1}{2}\alpha_{x}^{n+\frac{1}{2}}\right)u_{i,j}^{n+\frac{1}{2}}+\frac{1}{2}\beta^{n+\frac{1}{2}}u_{i+1,j}^{n+\frac{1}{2}}+\frac{1}{2}\gamma^{n+\frac{1}{2}}u_{i-1,j}^{n+\frac{1}{2}}=\left(1-\frac{1}{2}\alpha_{0}^{n}-\frac{1}{2}\alpha_{y}^{n}\right)u_{i,j}^{n}-\frac{1}{2}\zeta^{n}u_{i.j+1}^{n}-\frac{1}{2}\eta^{n}u_{i,j-1}^{n}
\]


where the dcoefficients are obtained by applying discretization schemes
on the PDEs $\left(\ref{eq:infPDE}\right)$ and $\left(\ref{eq:supPDE}\right)$.
In particular,
\begin{itemize}
\item $\alpha_{0}$ is the coefficient in front of $u_{i,j}$ coming from
the ``$u$'' term 
\item $\alpha_{x}$ is the coefficient in front of $u_{i,j}$ coming from
the ``$\partial_{x}u$'' and ``$\partial_{xx}u$'' terms
\item $\alpha_{y}$ is the coefficient in front of $u_{i,j}$ coming from
the ``$\partial_{y}u$'' term
\item $\beta$ is the coefficient in front of $u_{i+1,j}$ coming from the
``$\partial_{x}u$'' and ``$\partial_{xx}u$'' terms
\item $\gamma$ is the coefficient in front of $u_{i-1,j}$ coming from
the ``$\partial_{x}u$'' and ``$\partial_{xx}u$'' terms
\item $\zeta$ is the coefficient in front of $u_{i,j+1}$ coming from the
``$\partial_{y}u$'' 
\item $\eta$ is the coefficient in front of $u_{i,j-1}$ coming from the
``$\partial_{y}u$''
\end{itemize}
Note that for the ADI scheme, the coefficients of the tridiagonal
matrices depends on the unknows. The vectorized problem actually writes,
for each column $j$
\[
A\left(u_{\cdot,j}^{n+\frac{1}{2}}\right)u_{\cdot,j}^{n+\frac{1}{2}}=b\left(u_{\cdot,j}^{n+1}\right)
\]


where $A$ is a tridiagonal matrix. The solution $u_{\cdot,j}^{n+\frac{1}{2}}$
is found by iterating the following scheme until the matrix $A^{k}$
converges%
\footnote{The proof of the convergence is not covered here.%
}:
\[
\begin{cases}
A\left(u_{\cdot,j}^{n+\frac{1}{2},k-1}\right)u_{\cdot,j}^{n+\frac{1}{2},k} & =b\left(u_{\cdot,j}^{n+1}\right)\\
u_{\cdot,j}^{n+\frac{1}{2},0} & =u_{\cdot,j}^{n+1}
\end{cases}
\]


The same problem occurs in the second step and a similar algorithm
is used.


\section{Results}


\subsection*{NOTES}
\begin{itemize}
\item The presentation of the results must be more structured with the conclusion
at the end
\item The figures must be referenced
\item Notice that in the results presented below, I have plotted the price
in function of all the pairs $\left(\Gamma,\Phi\right)$ in the space
grid at time $0$. The graph is projected in a 2D space by mapping
the $\left(\Gamma,\Phi\right)$ to the LIBOR. In practice, at time
$0$, only one pair should be looked at: $\left(0,0\right)$. Looking
at the price in other coordinates is equivalent to look at the price
generated from the following zero curves
\[
f_{0}^{\left(\Gamma,\Phi\right)}\left(T\right)=f_{0}\left(T\right)+k\left(0,T\right)\Phi\gamma\left(0,T\right)+k\left(0,T\right)\Gamma
\]
These curves are NOT flat unless $\left(\Gamma,\Phi\right)=\left(0,0\right)$.
Hence, it does not make a lot of sense to compare the price obtained
from the different zero curve generated for the different pairs $\left(\Gamma,\Phi\right)$.
Instead, we should either bump the initial flat curve and reprice
the derivative or change the strike and only look at the point $\left(0,0\right)$
at time 0.
\end{itemize}

\subsection{Caplet}

In this section, we compared the upper bound price of a caplet with
maturity 1 year and strike 0 on the LIBOR with tenor $\delta$, obtained
from the PDE solver and the theoretical price. \\


Given the monotonicity of the caplet with the volatility parameter
$\sigma$ in the Hull-White model, the upper bound price corresponds
to the Hull-White price with the highest volatility. \\


As it can be seen in the figure below, the price obtained using the
PDE matches the close-form price. However, when comparing the corresponding
normal implied volatility, it can be seen that price obtained is not
accurate enough to reach the correct implied volatility, especially
far in-the-money. This will be investigated further. 

\begin{figure}[H]
\begin{centering}
\includegraphics[width=0.75\textwidth]{UVHJM/Caplet}
\par\end{centering}

\caption{Caplet maturity $T=1$ on the $\delta-$LIBOR rate. Model parameters:
$\sigma_{inf}=1\%$, $\sigma_{sup}=5\%$. Solver parameters: Number
of time points = 11, $x_{min}=x_{0}-\sigma_{max}\sqrt{T}$, $x_{max}=x_{0}+\sigma_{max}\sqrt{T}$,
$y_{min}=0$, $y_{max}=\sigma_{max}^{2}T$}


\end{figure}





\subsection{Binary Caplet}

In order to choose the number of discretization points in the ``space''
grid, we will compare the HW Binary caplet price with the one obtained
from the uncertain volatility non-linear PDE with $\sigma_{inf}=\sigma_{sup}$. 

In that degenerate case, the non-linear PDE is reduced to the linear
HW PDE. As it can be seen in the figures below, there is still room
for improvement to increase the accuracy of the PDE solver. 

\begin{figure}[H]
\begin{centering}
\includegraphics[width=0.75\textwidth]{UVHJM/BinaryCaplet}
\par\end{centering}

\caption{Price Binary Caplet with strike $K=0$ and maturity $T=1$ on the
LIBOR with tenor $\delta$ using the PDE or the close-form formula.
Solver parameters: Number of time points = 11, Number of space points
=41, $x_{min}=x_{0}-\sigma_{max}\sqrt{T}$, $x_{max}=x_{0}+\sigma_{max}\sqrt{T}$,
$y_{min}=0$, $y_{max}=\sigma_{max}^{2}T$}


\end{figure}




We now used the Uncertain Volatility non-linear PDE with $\sigma_{inf}=1\%$
and $\sigma_{sup}=5\%$ to retrieve the upper and the lower bound
of a binary caplet with maturity 1 year and strike 0 on the LIBOR
with tenor $\delta$.

Given the non-monotonicity of the Hull-white price of a binary option
with the volatility parameter, we expect the upper bound price to
be non-attainable by any prices obtained from a constant volatility
model with volatility between 1\% and 5\%. 

The same result is expected for the lower bound price. Moreover, the
prices of a binary option computed from a Hull-White model with constant
volatility between 1\% and 5\% should be between the upper and the
lower bound price obtained from the UV non-linear PDE.

\begin{figure}[H]
\begin{centering}
\includegraphics[width=0.75\textwidth]{UVHJM/BinaryCapletwithBounds}
\par\end{centering}

\caption{Price Binary Caplet with strike $K=0$ and maturity $T=1$ on the
LIBOR with tenor $\delta$. Solver parameters: Number of time points
= 11, Number of space points =41, $x_{min}=x_{0}-\sigma_{max}\sqrt{T}$,
$x_{max}=x_{0}+\sigma_{max}\sqrt{T}$, $y_{min}=0$, $y_{max}=\sigma_{max}^{2}T$}
\end{figure}


\newpage{}


\section{Appendix }


\subsection{Derivation of the HJM Framework\label{sub:DerivationHJM}}


\subsubsection{Instantaneous Forward Rate}

The instantaneous forward rate $f\left(t,T\right)$ with $t\leq T$
is the continuously compounded rate seen at time $t$ over the infinitesimal
period $\left[T,T+\delta T\right]$. It is defined with $B\left(t,T\right)$
the price of the zero-coupon bond at time $t$ and maturing at time
$T$.

\[
f\left(t,T\right)=\lim_{\delta T\rightarrow0}\frac{1}{\delta T}\ln\left[\frac{B\left(t,T+\delta T\right)}{B\left(t,T\right)}\right]
\]


or equivalently,
\begin{equation}
B\left(t,T\right)=e^{-\int_{t}^{T}f\left(t,u\right)\mathrm{d}u}\label{eq:forwardDef}
\end{equation}



\subsubsection{No-arbitrage assumption and implication}

In the HJM framework, the dynamics of the instantaneous forward rate
is governed by the no-arbitrage argument. Under the risk-neutral measure
$\mathbb{Q}$ (that exists given the no-arbitrage argument), the instantaneous
forward process can be expresed as follow:
\begin{equation}
\mathrm{d}f\left(t,T\right)=\mu_{f}\left(t,T\right)\mathrm{d}t+\sigma_{f}\left(t,T\right)\mathrm{d}W_{t}\label{eq:HJM}
\end{equation}


The drift $\mu_{f}$ in the equation $\left(\ref{eq:HJM}\right)$
must satisfies some condition for the no-arbitrage assumption to hold. 

In fact, under the risk-neutral measure, the instantaneous drift of
every tradable asset $A_{t}$ is given by $r_{t}A_{t}$. In particular,
this must be true for zero-coupon bonds. 

From the equation $\left(\ref{eq:forwardDef}\right)$, 
\begin{eqnarray}
\mathrm{d}\ln B\left(t,T\right) & = & -\mathrm{d}\int_{t}^{T}f\left(t,u\right)\mathrm{d}u\nonumber \\
 & = & r\left(t\right)\mathrm{d}t-\left[\int_{t}^{T}\mu_{f}\left(t,u\right)\mathrm{d}t+\int_{t}^{T}\sigma_{f}\left(t,u\right)\mathrm{d}W_{t}\right]\mathrm{d}u\label{eq:LogBondDiff}\\
 & = & \left[r\left(t\right)-\int_{t}^{T}\mu_{f}\left(t,u\right)\mathrm{d}u\right]\mathrm{d}t+\left[\int_{t}^{T}\sigma_{f}\left(t,u\right)\mathrm{d}u\right]\mathrm{d}W_{t}\nonumber 
\end{eqnarray}


In the equation $\left(\ref{eq:LogBondDiff}\right)$, the second line
is obtained by applying the Leibniz's rule for differenciation%
\footnote{See \url{https://en.wikipedia.org/wiki/Leibniz_integral_rule}%
}. Hence, applying the Ito lemma, we find:
\[
\frac{\mathrm{d}B\left(t,T\right)}{B\left(t,T\right)}=\left[r\left(t\right)-\int_{t}^{T}\mu_{f}\left(t,u\right)\mathrm{d}u+\frac{1}{2}\left(\int_{t}^{T}\sigma_{f}\left(t,u\right)\mathrm{d}u\right)^{2}\right]\mathrm{d}t+\left[\dots\right]\mathrm{d}W_{t}
\]


Under the risk-neutral measure, we must have:
\[
r\left(t\right)=r\left(t\right)-\int_{t}^{T}\mu_{f}\left(t,u\right)\mathrm{d}u+\frac{1}{2}\left(\int_{t}^{T}\sigma_{f}\left(t,u\right)\mathrm{d}u\right)^{2}
\]


This implies, when differenciating with respect to $T$:
\begin{equation}
\mu_{f}\left(t,T\right)=\sigma_{f}\left(t,T\right)\int_{t}^{T}\sigma_{f}\left(t,u\right)\mathrm{d}u\label{eq:noArbitrage}
\end{equation}


Hence, in the HJM framework, the dynamics of the instantaneous forward
rate is entirely defined by the term structure of the volatility $\sigma_{f}\left(t,T\right)$.


\subsubsection{Low-dimensional Markovian Representation of the Forward Curve}

In general, each point of the forward curve will be driven by its
own diffusion. It is possible to represent the forward curve with
a low-dimensional Markovian system when the volatility as the following
structure: 
\begin{equation}
\sigma_{f}\left(t,T\right)=\sigma_{f}\left(t,s\right)\exp\left(-\int_{s}^{T}\beta\left(u\right)\mathrm{d}u\right)\label{eq:separableCond}
\end{equation}


In \cite{RS}, the authors proven that this structure if a suffcient
condition for the forward curve to be represented with a low-dimensional
Markovian system. Another approach is presented to have an intuition
on the origin of the result.

The short-rate $r_{t}=f\left(t,t\right)$ can be expressed as follow
by integrating $\left(\ref{eq:HJM}\right)$ 
\[
r_{t}=f\left(0,t\right)+\int_{0}^{t}\sigma_{f}\left(x,t\right)\left(\int_{x}^{t}\sigma_{f}\left(x,u\right)\mathrm{d}u\right)\mathrm{d}x+\int_{0}^{t}\sigma_{f}\left(x,t\right)\mathrm{d}W_{x}
\]


The differential form is computed using the Leibnitz's rule:
\begin{equation}
\mathrm{d}r_{t}=\frac{\mathrm{d}f\left(0,t\right)}{\mathrm{d}t}\mathrm{d}t+\sigma_{t}\mathrm{d}W_{t}+\left[\int_{0}^{t}\frac{\partial\sigma_{f}}{\partial t}\left(x,t\right)\left(\int_{x}^{t}\sigma_{f}\left(x,u\right)\right)\mathrm{d}x+\int_{0}^{t}\sigma_{f}^{2}\left(x,t\right)\mathrm{d}x+\int_{0}^{t}\frac{\partial\sigma_{f}}{\partial t}\left(x,t\right)\mathrm{d}W_{x}\right]\mathrm{d}t\label{eq:shortRateDyn}
\end{equation}


Given that the drift of the short-rate has a path-dependent component,
it is not Markovian in general. However, if that path-dependent term
can be written as a linear combination of the short-term rate $r_{t}$
and another Markovian variable, then we would obtain a Markovian system.

Let us define 
\[
\Phi_{t}=\int_{0}^{t}\sigma_{f}^{2}\left(x,t\right)\mathrm{d}x
\]


Let us assume that there exists a deterministic function $\alpha$
such that 
\[
\int_{0}^{t}\frac{\partial\sigma_{f}}{\partial t}\left(x,t\right)\mathrm{d}W_{x}=\alpha\left(t\right)\int_{0}^{t}\sigma_{f}\left(x,t\right)\mathrm{d}W_{x}
\]


This yields 
\begin{eqnarray}
 &  & \frac{\partial\sigma_{f}}{\partial t}\left(x,t\right)=\alpha\left(t\right)\sigma_{f}\left(x,t\right)\nonumber \\
 & \Leftrightarrow & \sigma_{f}\left(x,T\right)=\sigma_{f}\left(x,t\right)\exp\left(\int_{t}^{T}\alpha\left(u\right)\mathrm{d}u\right)\label{eq:separabCond2}
\end{eqnarray}


Now, let us assume that the volatility is defined as in $\left(\ref{eq:separabCond2}\right)$.
Then the path-dependent drift become:
\begin{eqnarray*}
 &  & \int_{0}^{t}\frac{\partial\sigma_{f}}{\partial t}\left(x,t\right)\left(\int_{x}^{t}\sigma_{f}\left(x,u\right)\right)\mathrm{d}x+\int_{0}^{t}\sigma_{f}^{2}\left(x,t\right)\mathrm{d}x+\int_{0}^{t}\frac{\partial\sigma_{f}}{\partial t}\left(x,t\right)\mathrm{d}W_{x}\\
 & = & \int_{0}^{t}\alpha\left(t\right)\sigma_{f}\left(x,t\right)\left(\int_{x}^{t}\sigma_{f}\left(x,u\right)\right)\mathrm{d}x+\Phi_{t}+\int_{0}^{t}\alpha\left(t\right)\sigma_{f}\left(x,t\right)\mathrm{d}W_{x}\\
 & = & \alpha\left(t\right)r_{t}+\Phi_{t}
\end{eqnarray*}


The process of $r_{t}$ and $\Phi_{t}$ are then, respectively:
\[
\begin{cases}
\mathrm{d}r_{t} & =\left[\frac{\mathrm{d}f\left(0,t\right)}{\mathrm{d}t}+\alpha\left(t\right)r_{t}+\Phi_{t}\right]\mathrm{d}t+\sigma_{t}\mathrm{d}W_{t}\\
\mathrm{d}\Phi_{t} & =\left[\sigma_{t}^{2}+2\alpha\left(t\right)\Phi_{t}\right]\mathrm{d}t
\end{cases}
\]


Assuming the volatility $\sigma_{t}$ is at most a function of $r_{t}$
and $\Phi_{t}$, $\left(r_{t},\Phi_{t}\right)$ is indeed a Markovian
system. The sufficient condition $\left(\ref{eq:separabCond2}\right)$
is referred as the separability condition.

In the litterature, $r_{t}$ is replaced by $\Gamma_{t}$ and $\alpha\left(t\right)=-\beta\left(t\right)$
\[
\Gamma_{t}=\int_{0}^{t}\sigma_{f}\left(x,t\right)\left(\int_{x}^{t}\sigma_{f}\left(x,u\right)\mathrm{d}u\right)\mathrm{d}x+\int_{0}^{t}\sigma_{f}\left(x,t\right)\mathrm{d}W_{x}
\]


whose SDE is
\[
\mathrm{d}\Gamma_{t}=\left(\Phi_{t}-\beta\left(t\right)\Gamma_{t}\right)\mathrm{d}t+\sigma_{t}\mathrm{d}W_{t}
\]


It can then be noticed that this condition is also sufficient for
the forward curve to be represented by a two-dimensional Markovian
system:
\begin{eqnarray*}
f\left(t,T\right) & = & f\left(0,T\right)+\int_{0}^{t}\sigma_{f}\left(s,T\right)\left(\int_{s}^{T}\sigma_{f}\left(s,u\right)\mathrm{d}u\right)\mathrm{d}s+\int_{0}^{t}\sigma_{f}\left(s,T\right)\mathrm{d}W_{s}\\
 & = & f\left(0,T\right)+\int_{0}^{t}k\left(t,T\right)\sigma_{f}\left(s,t\right)\left(\int_{s}^{t}\sigma_{f}\left(s,u\right)\mathrm{d}u+\sigma_{f}\left(s,u\right)\int_{t}^{T}k\left(t,u\right)\mathrm{d}u\right)\mathrm{d}s+k\left(t,T\right)\int_{0}^{t}\sigma_{f}\left(s,t\right)\mathrm{d}W_{s}\\
 & = & f\left(0,T\right)+k\left(t,T\right)\Gamma_{t}+k\left(t,T\right)\Phi_{t}\gamma\left(t,T\right)
\end{eqnarray*}


with $k\left(t,T\right)=\exp\left(-\int_{t}^{T}\beta\left(s\right)\mathrm{d}s\right)$
and $\gamma\left(t,T\right)=\int_{t}^{T}k\left(t,s\right)\mathrm{d}s$.


\subsection{Derivation of the HJB Partial Differential Equation in the HJM Framework\label{sub:DerivationHJB}}


\subsection{Derivation of the Numerical Schemes\label{sub:DerivationScheme}}

The discretization schemes used to numerically solve the PDE are the
following:
\[
\partial_{t}u\left(t,x_{i},y_{j}\right)=\frac{u_{i,j}^{n+1}-u_{i,j}^{n}}{\delta t}
\]


\[
\partial_{x}u\left(t,x_{i},y_{j}\right)=\begin{cases}
\frac{u_{i+1,j}-u_{i,j}}{\delta x} & i<I\\
\frac{u_{i,j}-u_{i-1,j}}{\delta x} & i=I
\end{cases}
\]
\[
\partial_{y}u\left(t,x_{i},y_{j}\right)=\begin{cases}
\frac{u_{i,j+1}-u_{i,j}}{\delta y} & j<J\\
\frac{u_{i,j}-u_{i,j-1}}{\delta y} & j=J
\end{cases}
\]


\[
\partial_{xx}u\left(t,x_{i},y_{j}\right)=\begin{cases}
\frac{u_{i+1,j}-u_{i,j}}{\delta x} & i=0\\
\frac{u_{i+1,j}-2u_{i,j}+u_{i,j}}{\delta x^{2}} & 0<i<I\\
\frac{u_{i,j}-u_{i-1,j}}{\delta x} & i=I
\end{cases}
\]


\newpage{}
\begin{thebibliography}{1}
\bibitem{RS} P. Ritchken, L. Sankarasubramanian (1992) On Markovian
Representation of the Term Structure, Working papers of the Federal
Reserve Bank of Cleveland.\end{thebibliography}

\end{document}
