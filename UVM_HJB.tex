% Preview source code

%% LyX 2.0.6 created this file.  For more info, see http://www.lyx.org/.
%% Do not edit unless you really know what you are doing.
\documentclass[english]{article}
\usepackage[T1]{fontenc}
\usepackage[latin9]{inputenc}
\usepackage{geometry}
\geometry{verbose,tmargin=2cm,lmargin=2cm}
\usepackage{babel}
\usepackage{amsmath}
\usepackage{amssymb}
\usepackage{esint}
\usepackage[unicode=true,
 bookmarks=true,bookmarksnumbered=false,bookmarksopen=false,
 breaklinks=false,pdfborder={0 0 1},backref=false,colorlinks=false]
 {hyperref}
\hypersetup{
 pdfauthor={Vathana.S.Leang}}

\makeatletter
%%%%%%%%%%%%%%%%%%%%%%%%%%%%%% Textclass specific LaTeX commands.
\numberwithin{equation}{section}
\numberwithin{figure}{section}
\numberwithin{table}{section}

\makeatother

\begin{document}

\title{Uncertain Volatility in the HJM Framework}

\maketitle

\section{The HJM Framework}

In the HJM framework, under the risk-neutral measure $\mathbb{Q}$,
the instantaneous forward rate is defined as follow. For $t\leq T$:
\[
f\left(t,T\right)=f_{0}\left(T\right)+k\left(t,T\right)\Phi\left(t\right)\gamma\left(t,T\right)+k\left(t,T\right)\Gamma\left(t\right)
\]


where 
\begin{itemize}
\item $T\rightarrow f_{0}\left(T\right)$ is the zero curve with $f_{0}\left(0\right)$
being today's fixing
\item $k\left(t,T\right)=\exp\left(-\int_{t}^{T}\beta\left(s\right)\mathrm{d}s\right)$
and $\gamma\left(t,T\right)=\int_{t}^{T}k\left(t,s\right)\mathrm{d}s$
with $\beta_{t}$ being the mean-reversion parameter
\item $\left(\Gamma,\Phi\right)_{t}$ is a Markovian system given by 
\begin{equation}
\begin{cases}
\mathrm{d}\Gamma_{t} & =\sigma_{t}\mathrm{d}W_{t}^{\mathbb{Q}}+\left(\Phi_{t}-\beta_{t}\Gamma_{t}\right)\mathrm{d}t\\
\mathrm{d}\Phi_{t} & =\left(\sigma_{t}^{2}-2\beta_{t}\Phi_{t}\right)\mathrm{d}t\\
\left(\Gamma_{0},\Phi_{0}\right) & =\left(0,0\right)
\end{cases}\label{eq:Markov}
\end{equation}

\end{itemize}
The instantaneous forward rate is fully defined by the mean-reversion
$\beta_{t}$, the instantaneous volatility $\sigma_{t}$ of the short
rate and the zero curve $f_{0}$.


\subsection{Zero-Coupon Fixed rate Bond}

In th HJM framework, the price of zero-coupon fixed rate bond under
the risk-neutral measure $\mathbb{Q}$ is given by, for $t\leq T$:
\begin{eqnarray*}
B\left(t,T\right) & = & \exp\left[-\int_{t}^{T}f\left(t,s\right)\mathrm{d}s\right]\\
 & = & \exp\left[-\int_{t}^{T}f_{0}\left(s\right)\mathrm{d}s\right]\exp\left[-\left(\frac{1}{2}\Phi\left(t\right)\gamma^{2}\left(t,T\right)+\Gamma\left(t\right)\gamma\left(t,T\right)\right)\right]
\end{eqnarray*}


It can also be defined as follow
\begin{equation}
\begin{cases}
u^{bond}\left(t,x,y\right) & =\mathbb{E}_{t}^{\mathbb{Q}}\left[\exp\left(-\int_{t}^{T}f\left(s,s\right)\mathrm{d}s\right)|\Gamma\left(t\right)=x,\Phi\left(t\right)=y\right]\\
u^{bond}\left(T,x,y\right) & =1
\end{cases}\label{eq:bond}
\end{equation}


where $u^{bond}$ is the solution of the 2-dimensional PDE associated
with the system $\left(\ref{eq:Markov}\right)$ with the terminal
condition defined in $\left(\ref{eq:bond}\right)$.


\subsection{LIBOR rate and European Options indexed on the LIBOR rate}

Using the same notation defined in the previous section, the LIBOR
rate $L\left(t,T,T+\delta\right)$ with tenor $\delta$ for $t\leq T$
is defined by 
\[
L\left(t,T,T+\delta\right)=\frac{1}{\delta}\left[\frac{B\left(t,T\right)}{B\left(t,T+\delta\right)}-1\right]
\]


Considering a product paying a function $g$ of the LIBOR rate with
tenor $\delta$ fixed at time $T$ at $T+\delta$. 

Its value at time $t$ under the risk-neutral measure $\mathbb{Q}$
is given by
\begin{equation}
\begin{cases}
u\left(t,x,y\right) & =\mathbb{\mathbb{E}}_{t}^{\mathbb{Q}}\left[\exp\left(-\int_{t}^{T+\delta}f\left(s,s\right)\mathrm{d}s\right)g\left(L\left(T,T,T+\delta\right)\right)|\Gamma\left(t\right)=x,\Phi\left(t\right)=y\right]\\
u\left(T,x,y\right) & =g\left(L\left(T,T,T+\delta\right)\right)\cdot B\left(T,T+\delta\right)\\
u\left(T+\delta,x,y\right) & =g\left(L\left(T,T,T+\delta\right)\right)
\end{cases}\label{eq:LIBORderivative}
\end{equation}


where $u$ is the solution of the 2-dimensional PDE associated with
the system $\left(\ref{eq:Markov}\right)$ with the terminal condition
defined in $\left(\ref{eq:LIBORderivative}\right)$.


\section{Adding Uncertain Volatility}

As stated in the previous section, the instantaneous forward curve
as well as the pricing measure $\mathbb{Q}$ are dependent on the
choice of the triplet $\left(\beta_{t},\sigma_{t},f_{0}\right)$.\\


From now on, the mean-reversion $\beta_{t}$ is supposed to be constant
and hence, a pricing measure $\mathbb{Q}^{\sigma_{t}}$ is associated
to a given volatility process $\sigma_{t}$. Let us notice that one
volatility process can be associated to more than one pricing measure
(if $\sigma_{t}$ is driven by its own Brownian motion for example).\\


Adding uncertain volatility in the HJM framework consists in considering
all the possible volatility processes taking values in a given range
$\left[\sigma_{inf},\sigma_{sup}\right]$ and computing the highest
and the lowest possible prices of a product given all the possible
associated pricing measures. Formally:
\begin{itemize}
\item Let $\mathcal{F}_{\left[\sigma_{inf},\sigma_{sup}\right]}^{\left[0,T\right]}$
be the set of all volatility processes $\sigma_{t}$ taking values
in the range $\left[\sigma_{inf},\sigma_{sup}\right]$ for each $t\in\left[0,T\right]$.
\item Let $\mathcal{Q}_{\left[\sigma_{inf},\sigma_{sup}\right]}^{\left[0,T\right]}$
be the set of all risk-neutral measures associated to all $\sigma_{t}\in\mathcal{F}_{\left[\sigma_{inf},\sigma_{sup}\right]}^{\left[0,T\right]}$
\item Considering a product paying a function $g$ of the LIBOR rate with
tenor $\delta$ fixed at time $T$ at $T+\delta$, the highest and
the lowest possible prices are defined by
\begin{equation}
\begin{cases}
u_{sup}\left(t,x,y\right) & =\sup_{\mathbb{Q}\in\mathcal{Q}_{\left[\sigma_{inf},\sigma_{sup}\right]}^{\left[0,T+\delta\right]}}\mathbb{E}_{t}^{\mathbb{Q}}\left[\exp\left(-\int_{t}^{T+\delta}f\left(s,s\right)\mathrm{d}s\right)g\left(L\left(T,T,T+\delta\right)\right)|\Gamma\left(t\right)=x,\Phi\left(t\right)=y\right]\\
u_{sup}\left(T+\delta,x,y\right) & =g\left(L\left(T,T,T+\delta\right)\right)
\end{cases}\label{eq:supPrice}
\end{equation}
\begin{equation}
\begin{cases}
u_{inf}\left(t,x,y\right) & =\inf_{\mathbb{Q}\in\mathcal{Q}_{\left[\sigma_{inf},\sigma_{sup}\right]}^{\left[0,T+\delta\right]}}\mathbb{E}_{t}^{\mathbb{Q}}\left[\exp\left(-\int_{t}^{T+\delta}f\left(s,s\right)\mathrm{d}s\right)g\left(L\left(T,T,T+\delta\right)\right)|\Gamma\left(t\right)=x,\Phi\left(t\right)=y\right]\\
u_{inf}\left(T+\delta,x,y\right) & =g\left(L\left(T,T,T+\delta\right)\right)
\end{cases}\label{eq:infPrice}
\end{equation}

\end{itemize}
It can be shown that $u_{inf}$ and $u_{sup}$ respectively satisfies
the following non-linear PDEs
\begin{equation}
\partial_{t}u+\left(y-\beta x\right)\cdot\partial_{x}u-2\beta\cdot y\cdot\partial_{y}u+\frac{1}{2}\left(\partial_{xx}u+2\partial_{y}u\right)\cdot\Sigma_{inf}^{2}\left(\partial_{xx}u+2\partial_{y}u\right)-\left(f_{0}\left(t\right)+x\right)\cdot u=0\label{eq:infPDE}
\end{equation}


with 
\[
\Sigma_{inf}\left(x\right)=\begin{cases}
\sigma_{inf} & x\ge0\\
\sigma_{sup} & x<0
\end{cases}
\]


\begin{equation}
\partial_{t}u+\left(y-\beta x\right)\cdot\partial_{x}u-2\beta\cdot y\cdot\partial_{y}u+\frac{1}{2}\left(\partial_{xx}u+2\partial_{y}u\right)\cdot\Sigma_{sup}^{2}\left(\partial_{xx}u+2\partial_{y}u\right)-\left(f_{0}\left(t\right)+x\right)\cdot u=0\label{eq:supPDE}
\end{equation}


with 
\[
\Sigma_{sup}\left(x\right)=\begin{cases}
\sigma_{sup} & x\ge0\\
\sigma_{inf} & x<0
\end{cases}
\]



\section{Numerical Schemes}

Two discretization schemes will be used to solve the PDEs $\left(\ref{eq:infPDE}\right)$
and $\left(\ref{eq:supPDE}\right)$: the explicit scheme and the ADI
scheme.

The grid $\left(t,x,y\right)$ is assumed to be ``isotropic'' in
the sense that the x-axis is independent from coordinate $\left(t,y\right)$.
Same along the y-axis and the t-axis.

The partition of each axis is assumed to be regular for now. For $\left(i,j,n\right)\in\left[0,\dots,I\right]\times\left[0,\dots J\right]\times\left[0,\dots,N\right]$
\begin{eqnarray*}
x_{i} & = & x_{0}+i\cdot\delta x\\
y_{j} & = & y_{0}+j\cdot\delta y\\
t^{n} & = & t^{0}+n\cdot\delta t
\end{eqnarray*}
for

The price $u\left(t,x,y\right)$ and its time and spatial derivatives
are discretized as follow
\[
u\left(t^{n},x_{i},y_{j}\right)=u_{i,j}^{n}
\]


\[
\partial_{t}u\left(t^{n},x_{i},y_{j}\right)=\frac{u_{i,j}^{n+1}-u_{i,j}^{n}}{\delta t}
\]


\[
\partial_{x}u\left(t^{n},x_{i},y_{j}\right)=\begin{cases}
\frac{u_{i+1,j}^{n}-u_{i,j}^{n}}{\delta x} & i<I\\
\frac{u_{i,j}^{n}-u_{i-1,j}^{n}}{\delta x} & i=I
\end{cases}
\]
\[
\partial_{y}u\left(t^{n},x_{i},y_{j}\right)=\begin{cases}
\frac{u_{i,j+1}^{n}-u_{i,j}^{n}}{\delta y} & j<J\\
\frac{u_{i,j}^{n}-u_{i,j-1}^{n}}{\delta y} & j=J
\end{cases}
\]


\[
\partial_{xx}u\left(t^{n},x_{i},y_{j}\right)=\begin{cases}
\frac{u_{i+1,j}^{n}-u_{i,j}^{n}}{\delta x} & i=0\\
\frac{u_{i+1,j}^{n}-2u_{i,j}^{n}+u_{i,j}^{n}}{\delta x^{2}} & 0<i<I\\
\frac{u_{i,j}^{n}-u_{i-1,j}^{n}}{\delta x} & i=I
\end{cases}
\]



\paragraph{Explicit Scheme}

Solve backward from $N$ to $0$, for each $\left(i,j\right)$

\begin{eqnarray*}
u_{i,j}^{n} & = & u_{i,j}^{n+1}-\alpha^{0}u_{i,j}^{n+1}-\alpha^{x}u_{i,j}^{n+1}-\alpha^{y}u_{i,j}^{n+1}-\beta u_{i+1.j}^{n+1}-\gamma u_{i-1,j}^{n+1}-\zeta u_{i,j+1}^{n+1}-\eta u_{i,j-1}^{n+1}
\end{eqnarray*}



\paragraph{ADI Scheme}

Solve the PDE backwardly from $N$ to $0$ in two steps. 
\begin{enumerate}
\item The scheme is explicit in the y-axis and implicit in the x-axis. For
each ``column'' $j$, we have an equation of the form $Ax=b$ where
$A$ is a tridiagonal matrix. We solve it to find the intermediate
$u^{n+\frac{1}{2}}$ price.
\[
\left(1-\frac{1}{2}\alpha^{0}-\frac{1}{2}\alpha^{y}\right)u_{i,j}^{n+1}-\frac{1}{2}\zeta u_{i,j+1}^{n+1}-\frac{1}{2}\eta u_{i,j-1}^{n+1}=\left(1+\frac{1}{2}\alpha^{x}\right)u_{i,j}^{n+\frac{1}{2}}+\frac{1}{2}\beta u_{i+1.j}^{n+\frac{1}{2}}+\frac{1}{2}\gamma u_{i-1,j}^{n+\frac{1}{2}}
\]

\item The scheme is implicit in the y-axis and explicit in the y-axis. For
each ``row'' $i$, we have an equation of the form $Ax=b$ where
$A$ is a tridiagonal matrix. We solve it to find the intermediate
$u^{n}$ price.
\end{enumerate}
\[
\left(1-\frac{1}{2}\alpha^{x}\right)u_{i,j}^{n+\frac{1}{2}}-\frac{1}{2}\beta u_{i+1,j}^{n+\frac{1}{2}}-2\gamma u_{i-1,j}^{n+\frac{1}{2}}=\left(1+\frac{1}{2}\alpha^{0}+\frac{1}{2}\alpha^{y}\right)u_{i,j}^{n}+\frac{1}{2}\zeta u_{i.j+1}^{n}+\frac{1}{2}\eta u_{i,j-1}^{n}
\]


where:
\begin{itemize}
\item $\alpha_{0}$ is the coefficient in front of $u_{i,j}$ coming from
the ``$u$'' term 
\item $\alpha_{x}$ is the coefficient in front of $u_{i,j}$ coming from
the ``$\partial_{x}u$'' term
\item $\alpha_{y}$ is the coefficient in front of $u_{i,j}$ coming from
the ``$\partial_{y}u$'' term
\item $\beta$ is the coefficient in front of $u_{i+1,j}$ coming from the
``$\partial_{x}u$'' and ``$\partial_{xx}u$'' terms
\item $\gamma$ is the coefficient in front of $u_{i-1,j}$ coming from
the ``$\partial_{x}u$'' and ``$\partial_{xx}u$'' terms
\item $\zeta$ is the coefficient in front of $u_{i,j+1}$ coming from the
``$\partial_{y}u$'' 
\item $\eta$ is the coefficient in front of $u_{i,j-1}$ coming from the
``$\partial_{y}u$''
\end{itemize}
For the explicit scheme, at step $n$, the coefficients $\alpha_{0}$,$\alpha_{x}$,
$\alpha_{y}$, $\beta$, $\gamma$, $\zeta$, $\eta$ are computed
using information from step $n+1$ and are dependent from the spatial
position $\left(i,j\right)$.

For the ADI scheme, In the first step, the coefficients $\alpha_{y}$
is computed from $u^{n+1}$ terms while $\beta$ and $\gamma$ are
computed using $u^{n+\frac{1}{2}}$ terms that are unknown.

The vectorized problem writes, for each column $j$
\[
A\left(u_{\cdot,j}^{n+\frac{1}{2}}\right)u_{\cdot,j}^{n+\frac{1}{2}}=b\left(u_{\cdot,j}^{n+1}\right)
\]


where $A$ is the tridiagonal matrix. The solution $u_{\cdot,j}^{n+\frac{1}{2}}$
is found by iterating the following scheme until the matrix $A^{k}$
converges:
\[
\begin{cases}
A\left(u_{\cdot,j}^{n+\frac{1}{2},k-1}\right)u_{\cdot,j}^{n+\frac{1}{2},k} & =b\left(u_{\cdot,j}^{n+1}\right)\\
u_{\cdot,j}^{n+\frac{1}{2},0} & =u_{\cdot,j}^{n+1}
\end{cases}
\]


The same problem occurs in the second step and a similar algorithm
is used.
\end{document}
