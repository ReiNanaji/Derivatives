% Preview source code

%% LyX 2.0.6 created this file.  For more info, see http://www.lyx.org/.
%% Do not edit unless you really know what you are doing.
\documentclass[english]{article}
\usepackage[T1]{fontenc}
\usepackage[latin9]{inputenc}
\usepackage{geometry}
\geometry{verbose,lmargin=2cm,rmargin=2cm}
\usepackage{amsmath}
\usepackage{amssymb}
\usepackage{babel}
\begin{document}

\title{Notes on Uncertain Volatility Model}

\maketitle

\section{Model Specification}


\subsection{Volatility Modeling}

NO assumption is made on the volatility distribution. The only prior
knowledge that is injected is the volatility band to which the realized
volatility is going to belong to. This can be determined with historical
data or implied volatility data. 

The model is given by:

\begin{equation}
\frac{\mathrm{d}S_{t}}{S_{t}}=\sigma_{t}\mathrm{d}Z_{t}
\end{equation}


where $\sigma_{t}\in\left[\sigma_{min},\sigma_{max}\right]\forall t\in\left[0,T\right]$.


\subsection{Pricing under the UVM}

Since we are considering ALL the volatility processes whose support
is in the volatility band, MANY martingale measures exists. A unique
price makes no sense in the UVM. Instead we can define the seller
and the buyer price which respectively corresponds to super-replication
and the sub-replication price. In general, 


\paragraph{Buyer's price}

Highest price that someone agrees to pay such that there exists a
dynamic hedging strategy ensuring that the hedged portfolio is positive
almost sure. This also corresponds to:

\begin{equation}
\mathcal{B}_{t}\left(S_{T}\right)=\inf_{\mathbb{P}\in ELMM}\mathbb{E}^{\mathbb{P}}\left[D_{tT}F\left(S_{T}\right)\right]\label{eq:2}
\end{equation}


where ``ELMM'' stands for equivalent local martingale measure. $D_{tT}$
is the discount factor from $T$ to $t$. $F$ is a payoff function
(vanilla in this case, for simplification).


\paragraph{Seller's price}

Lowest price at which someone agrees to sell such that there exists
a dynamic hedging strategy ensuring that the hedged portfolio is positive
almost sure. This also corresponds to:

\begin{equation}
\mathcal{S}_{t}\left(S_{T}\right)=\sup_{\mathbb{P}\in ELMM}\mathbb{E}^{\mathbb{P}}\left[D_{tT}F\left(S_{T}\right)\right]\label{eq:3}
\end{equation}


In the UVM context, for each measure $\mathbb{P}$, is associated
a volatility scenario and Equation \ref{eq:2} and \ref{eq:3} rewrites:

\begin{eqnarray}
u_{t}=\mathcal{B}_{t}\left(S_{T}\right) & = & \inf_{\sigma_{t}\in\mathcal{A}_{tT}}\mathbb{E}^{\mathbb{P}_{\sigma_{t}}}\left[D_{tT}F\left(S_{T}\right)\right]\\
v_{t}=\mathcal{S}_{t}\left(S_{T}\right) & = & \sup_{\sigma_{t}\in\mathcal{A}_{tT}}\mathbb{E}^{\mathbb{P}_{\sigma_{t}}}\left[D_{tT}F\left(S_{T}\right)\right]
\end{eqnarray}


where $\mathcal{A}_{tT}$ corresponds to the set of adapted volatility
process whose support is in the volatility band. In essence, the price
computed under the UVM corresponds to the worst case and the best
case scenario volatility and determine the band of no-arbitrage price. 


\subsection{Stochastic Control Problem: Black-Scholes Barenblatt equation}

Following from the Stochastic Control framework, we can link the seller's
and the buyer's price to non-linear PDE called HJB PDE. 

\begin{eqnarray*}
\partial_{t}u\left(t,x\right)+\frac{1}{2}\sup_{\sigma\in\left[\sigma_{min},\sigma_{max}\right]}\sigma^{2}x^{2}\partial_{x}^{2}u\left(t,x\right) & = & 0\\
u\left(T,x\right) & = & F\left(x\right)
\end{eqnarray*}


and

\begin{eqnarray*}
\partial_{t}v\left(t,x\right)+\frac{1}{2}\inf_{\sigma\in\left[\sigma_{min},\sigma_{max}\right]}\sigma^{2}x^{2}\partial_{x}^{2}v\left(t,x\right) & = & 0\\
v\left(T,x\right) & = & F\left(x\right)
\end{eqnarray*}


Or equivalently, 

\begin{eqnarray*}
\partial_{t}u\left(t,x\right)+\frac{1}{2}\Sigma^{2}\left(\partial_{x}^{2}u\left(t,x\right)\right)x^{2}\partial_{x}^{2}u\left(t,x\right) & = & 0\\
u\left(T,x\right) & = & F\left(x\right)
\end{eqnarray*}


and

\begin{eqnarray*}
\partial_{t}v\left(t,x\right)+\frac{1}{2}\Lambda^{2}\left(\partial_{x}^{2}u\left(t,x\right)\right)x^{2}\partial_{x}^{2}v\left(t,x\right) & = & 0\\
v\left(T,x\right) & = & F\left(x\right)
\end{eqnarray*}


where

\[
\Sigma\left(X\right)=\begin{cases}
\sigma_{max} & X\geq0\\
\sigma_{min} & X<0
\end{cases}
\]


\[
\Lambda\left(X\right)=\begin{cases}
\sigma_{min} & X\geq0\\
\sigma_{max} & X<0
\end{cases}
\]



\section{Calibration Complex: Lagrangian Uncertain Volatility Model}

For parametric models (Black-Scholes, SV, LV...), the common (does
not mean good) practice is to calibrate the model parameters to the
market smile. The initial ``motivation'' behind this practice is
related to the foundation of relative pricing and static replication.
The parameters are chosen such that the model mathes the price of
the vanilla instruments that statically replicate the exotic options.
Therefore, the exotic price are computed in the model using the information
contained in the vanilla price. This practice is missleading as the
implied dynamics may not capture the risk factors driving the exotic
price. A second point is that the information used for the calibration
may not ALL be relevant for the hedging of the exotic options. It
is equivalent to compute the projection of the price on a space that
is a bad approximation of the real price. However, the model won't
give ANY signal to the user concerning that matters. \\


One of the most stricking example may be the Local Volatility model
which is the unique one-factor model that perfectly matches the market
smile. Once the calibration is done, the user have no control whatsoever
on the dynamics of the smile. Hence, path-dependent product may be
mispriced in that model with no remediation. \\


In the UVM, the only two parameters are the boundary of the volatility
band. They are set using historical or implied data. The model is
not meant to match any volatility smile information. In \cite{key-1},
a model is constructed in order to incorporate market information
in a consistent and relevant way. The value of an exotic in that model
can be decomposed in two parts: the one corresponding to the projection
on the vanilla instruments and a second part corresponding to the
residue that is replicated dynamically under the UVM worst (or best
case scenario). The formal price is given by:

\begin{equation}
V\left(S_{T},t\right)=\inf_{\lambda_{1},\dots,\lambda_{m}}\left\{ \sup_{\sigma_{t}\in\mathcal{A}_{tT}}\mathbb{E}^{\mathbb{P}_{\sigma_{t}}}\left[F\left(S_{T}\right)-\sum_{i=1}^{M}\lambda_{i}e^{-r\left(t_{i}-t\right)}G_{i}\left(S_{t_{i}}\right)\right]+\sum_{i=1}^{M}\lambda_{i}C_{i}\right\} \label{eq:6}
\end{equation}


where $\lambda_{i}$ is the position in the $i^{th}$ vanilla instruments
with market price $C_{i}$ and payoff $G_{i}$ maturing at time $t_{i}$.

In essence, the algorithm will select the portfolio of vanilla instruments
that yields the ``best'' replication of the exotic options. This
is done by choosing the position $\lambda_{i}$ that minimizes the
contribution of the ``residual liability'' computed under the UVM
framework. Hence, solving this optimization problem will provide the
best vanilla portfolio and a residue that is valued under the worst
(best) scenario volatility. 

The model directly use the information from the hedging instrument
instead of using then the adjust the parameters. Moreover, we see
this extension as a form of ``calibration'' as Equation \ref{eq:6}
is equivalent to: find $\lambda_{1},\dots,\lambda_{M}$ such that:

\begin{eqnarray}
C_{i} & = & \mathbb{E}^{\tilde{\mathbb{P}}}\left[e^{-r\left(t_{i}-t\right)}G_{i}\left(S_{t_{i}}\right)\right]\\
\tilde{\mathbb{P}} & = & \arg\sup_{\sigma_{t}\in\mathcal{A}_{tT}}\mathbb{E}^{\mathbb{P}_{\sigma_{t}}}\left[F\left(S_{T}\right)-\sum_{i=1}^{M}\lambda_{i}e^{-r\left(t_{i}-t\right)}G_{i}\left(S_{t_{i}}\right)\right]
\end{eqnarray}


This can be interpreted as: the algorithm try to find the position
$\lambda_{i}$ such that the measure associated with the worst case
scenario also match the market price. 


\paragraph{Note 1}

We can check the algorithm by pricing a call or a digital option. 


\paragraph{Note 2}

The optimization problem will always a unique solution if it exists.
In fact, the supremum of linear function is convex.


\section{Numerical Implementation}


\subsection{Tree-based approach}

Remember that the UVM is not complete as the distribution of the volatility
is not specified and there is not a unique martingale measure but
a family of measure verying the condition on the volatility. Therefore,
binomial tree are not a suitable numerical approach. However, trinomial
trees may be a good candidate as it is an incomplete model. In fact,
given the no-arbitrage constraints, the risk-neutral probability is
not uniquely determined. We note that due to the non-uniqueness of
the risk-neutral measure, the volatility of the underlying is not
fixed. 


\paragraph*{Problem}

Find $P_{U}$, $P_{M}$ and $P_{D}$ and the ``good'' state parametrization
$U$, $M$ and $D$ such that 
\begin{enumerate}
\item $\ln\frac{S_{\delta t}}{S_{0}}$ is asymptotically normally distributed
with instantaneous mean $\mu$ and instantaneous variance $\sigma^{2}$. 
\item The maximum and the minimum volatility $\sigma$ achieved can be controlled
by the probability measure
\item The no-arbitrage constraint is satisfied
\begin{equation}
P_{D}D+P_{M}M+P_{D}D=e^{r\delta t}\label{eq:9}
\end{equation}

\item The branches recombine.
\end{enumerate}

\paragraph*{Parametrization of the jump}

In \cite{key-2}, 

\begin{eqnarray}
U & = & e^{\mu\delta t+u\sqrt{\delta t}}\\
M & = & e^{\mu\delta t+m\sqrt{\delta t}}\\
D & = & e^{\mu\delta t+d\sqrt{\delta t}}
\end{eqnarray}


And the coefficient $m$, $u$ and $d$ have to be specified. 


\paragraph*{Recombination condition}

This implies:

\[
UD=M^{2}
\]


Or equivalently, 

\[
m=\frac{u+d}{2}
\]



\paragraph{Asymptotic distribution }

Let us denote $P_{U}^{0}$, $P_{D}^{0}$ and $P_{M}^{0}$ the asymptotic
probability. We must have:

\begin{eqnarray}
\mathbb{E}\left[\ln\frac{S_{T}}{S_{0}}\right] & = & \mu\delta t\label{eq:13}\\
\mathbb{V}ar\left[\ln\frac{S_{T}}{S_{0}}\right] & = & \sigma^{2}\delta t\label{eq:14}
\end{eqnarray}


This gives:

\begin{eqnarray}
P_{D}^{0}d+P_{M}^{0}\left(\frac{u+d}{2}\right)+P_{U}^{0}u & = & 0\\
P_{D}^{0}d^{2}+P_{M}^{0}\left(\frac{u+d}{2}\right)^{2}+P_{U}^{0}u^{2} & = & \sigma^{2}
\end{eqnarray}


Equation \ref{eq:13} is a constraint where Equation \ref{eq:14}
can be seen as a computation of the variance delivered by the resulting
distribution. It is NOT a constraint !

Combining the constraint \ref{eq:13} with:

\[
P_{D}^{0}+P_{M}^{0}+P_{U}^{0}=1
\]


We have 3 unknows and 2 constraints. This leaves us with a degree
of freedom and we set $P_{D}^{0}=p$.

\begin{eqnarray*}
P_{U}^{0} & = & p-\frac{u+d}{u-d}\\
P_{M}^{0} & = & -2p+\frac{2u}{u-d}
\end{eqnarray*}


Since probabilities are bounded between 0 and 1, the widest interal
where $p$ can take values is given by:

\begin{equation}
\max\left\{ 0,\frac{u+d}{u-d}\right\} \leq p\leq\frac{u}{u-d}\label{eq:17}
\end{equation}


Note that the delivered volatility given by Equation \ref{eq:14}
is comprised between

\begin{eqnarray*}
\sigma_{max}^{2} & = & \frac{1}{2}p_{max}\left[\left(u-d\right)^{2}-u\left(u+d\right)\right]\\
\sigma_{min}^{2} & = & \frac{1}{2}p_{min}\left[\left(u-d\right)^{2}-u\left(u+d\right)\right]
\end{eqnarray*}


This highest and lowest volatility are defined by:

\begin{eqnarray}
\sigma_{max}^{2} & = & -du\\
\sigma_{min}^{2} & = & -mu\mathbf{1}_{m\leq0}-md\mathbf{1}_{m>0}
\end{eqnarray}



\paragraph{Martingale measure}

Equation \ref{eq:9} now writes with $\mu=r$

\begin{equation}
e^{r\delta t}=e^{r\delta t+u\sqrt{\delta t}}P_{U}+e^{r\delta t+\left(\frac{u+d}{2}\right)\sqrt{\delta t}}P_{M}+e^{r\delta t+d\sqrt{\delta t}}P_{D}\label{eq:20}
\end{equation}


Given the previous asymptotic distribution derived, we can parametrized
the distribution as follow

\begin{eqnarray*}
P_{U} & = & P_{U}^{0}+\pi_{u}\sqrt{\delta t}+o\left(\delta t\right)\\
P_{M} & = & P_{M}^{0}+\pi_{m}\sqrt{\delta t}+o\left(\delta t\right)\\
P_{D} & = & P_{D}^{0}+\pi_{d}\sqrt{\delta t}+o\left(\delta t\right)
\end{eqnarray*}


An asymptotic development of Equation \ref{eq:20} combined with a
unicity argument gives:

\begin{eqnarray*}
\pi_{u}+\pi_{m}+\pi_{d} & = & 0\\
\frac{1}{2}\left(P_{D}^{0}d^{2}+P_{M}^{0}\left(\frac{u+d}{2}\right)^{2}+P_{U}^{0}u^{2}\right)+u\pi_{u}+\left(\frac{u+d}{2}\right)\pi_{m}+d\pi_{d} & = & 0
\end{eqnarray*}


Here again the problem is overparametrized. However, we are already
using $p$ to control our probability measure, we can arbitrarily
set $\pi_{m}=0$. This gives to 

\begin{eqnarray*}
\pi_{u} & = & \frac{1}{2\left(d-u\right)}\left(P_{D}^{0}d^{2}+P_{M}^{0}\left(\frac{u+d}{2}\right)^{2}+P_{U}^{0}u^{2}\right)\\
\pi_{d} & = & -\frac{1}{2\left(d-u\right)}\left(P_{D}^{0}d^{2}+P_{M}^{0}\left(\frac{u+d}{2}\right)^{2}+P_{U}^{0}u^{2}\right)
\end{eqnarray*}



\paragraph{Attainable volatility and Choice of the price state}

Finally, the trinomial is entirely determined by the parameter $p$
and the parameters $u$ and $d$. In \cite{key-2}, the choice is
$u=-d=\sigma_{max}$. This gives $m=0$. On top of that, given that
the volatility band is chosen, the corresponding $p$ band is:

\[
\frac{1}{2}\left(\frac{\sigma_{min}}{\sigma_{max}}\right)^{2}\leq p\leq\frac{1}{2}
\]


Hence, the probability are defined as:

\begin{eqnarray*}
P_{U} & = & p\left(1-\frac{1}{2}\sigma_{max}\sqrt{\delta t}\right)+o\left(\delta t\right)\\
P_{M} & = & 1-2p+o\left(\delta t\right)\\
P_{D} & = & p\left(1+\frac{1}{2}\sigma_{max}\sqrt{\delta t}\right)+o\left(\delta t\right)
\end{eqnarray*}



\paragraph{Backward computation of the option value}

In the UVM context, the value of $p$ is monitored by the discrete
version of the gamma. 


\subsection{PDE }


\subsection{Monte-Carlo Method}


\section{Tests}


\subsection{Cliquet}

See \cite{key-4}. Cliquet option are exotic option which are basically
a series of forward starting call spread options. Cliquet options
are driven by the level of the forward skew. Constant model and deterministic
volatility are not suited as they do not provide a control on the
smile dynamics. In \cite{key-4}, two different approaches are tested
to value cliquet options.
\begin{itemize}
\item Consider that the volatility is constant but unknown with the level
between two boundary values
\item Consider that the volatility is unknown between two boundary values
but not necessarily constant. 
\end{itemize}
In the first situation, vega is a metric to mesure the model-delivered
price sensitivity with the constant volatility. In the second case,
the uncertainty on the price resulting from the uncertainty on the
volatility corresponds to the no-arbitrage band. What we see is that
the cliquet price is weakly sensitive to the constant volatility while
the price band in the UVM is much larger...
\begin{thebibliography}{1}
\bibitem{key-1}Managing The Volatility Risk of Portfolios of derivative
securities: the Lagrangian Uncertain Volatility model, Marco Avelleneda
and Antonio Paras

\bibitem{key-2}A new approach for pricing derivative securities in
markets with uncertain volatilities: A `` case study'' on the trinomial
tree, Arnon Levy, Avellanada and Antonio Paras

\bibitem{key-3}The Uncertain Volatility Model, Claude Martini, Antoine
Jacquier

\bibitem{key-4}Cliquet Options and Volatility Models, Paul Wilmott

\bibitem{key-5}Uncertain Volatility Model: A Monte-Carlo Approach,
Julien Guyon, Pierre Henry-Labordere\end{thebibliography}

\end{document}
