% Preview source code

%% LyX 2.0.6 created this file.  For more info, see http://www.lyx.org/.
%% Do not edit unless you really know what you are doing.
\documentclass[english]{article}
\usepackage[T1]{fontenc}
\usepackage[latin9]{inputenc}
\usepackage{geometry}
\geometry{verbose,tmargin=2.5cm,bmargin=2.5cm,lmargin=2.5cm,rmargin=2.5cm,headheight=31.941cm}
\usepackage{color}
\usepackage{babel}
\usepackage{float}
\usepackage{amsmath}
\usepackage[unicode=true,
 bookmarks=true,bookmarksnumbered=false,bookmarksopen=false,
 breaklinks=false,pdfborder={0 0 1},backref=false,colorlinks=false]
 {hyperref}
\hypersetup{
 pdfauthor={Vathana.S.Leang}}

\makeatletter

%%%%%%%%%%%%%%%%%%%%%%%%%%%%%% LyX specific LaTeX commands.
%% Because html converters don't know tabularnewline
\providecommand{\tabularnewline}{\\}

%%%%%%%%%%%%%%%%%%%%%%%%%%%%%% Textclass specific LaTeX commands.
\numberwithin{equation}{section}
\numberwithin{figure}{section}
\numberwithin{table}{section}

%%%%%%%%%%%%%%%%%%%%%%%%%%%%%% User specified LaTeX commands.
\usepackage[T1]{fontenc} % MiKTeX as ltxbase
\usepackage[latin9]{inputenc} % MiKTeX as ltxbase
\usepackage{lmodern} % for modern T1 fonts

% hyphenation patterns for English


% euro symbol
\usepackage{eurosym} % MiKTeX as eurosym

% used to prevent footnotes from disapearing when placed inside tables/other
\usepackage{footnote} % MiKTeX as mdwtools

% remove the default identation of new paragraphs
\usepackage{parskip} % MiKTeX as parskip


% for cross references
%\usepackage{hyperref} % MiKTeX as hyperref
\usepackage{bookmark} % MiKTeX as oberdiek

% for math
\usepackage{amsmath,amssymb,amsthm} % MiKTeX as amsmath, amscls
\usepackage{thmtools,thm-restate} % MiKTeX as thmtools

% for functions defined by branches
\usepackage{cases} % MiKTeX as cases

% for double stroke mathematical symbols
\usepackage{dsfont} % MiKTeX as dstroke

% to allow the use of appendices
\usepackage[titletoc,toc,title]{appendix} % MiKTeX as appendix

% to allow fixing a float in the place where it is declared
\usepackage{float} % MiKTeX as float

% used to color and highlight text. Used in the definition of \ins, \del, \hlt.
\usepackage[usenames,dvipsnames]{xcolor} % MiKTeX as xcolor

% insert comments in the PDF. Used in the definition of \cmt.
\usepackage{pdfcomment} % MiKTeX as pdfcomment, bezos

% decorate text. Used in the definition of \ins and \del.
\usepackage[normalem]{ulem} % MiKTeX as ulem

% makes figure and table captions prettier
\usepackage[font=small,labelfont=bf,margin=0.4cm]{caption} % MiKTeX as caption

% to include figures in the text
\usepackage{graphicx} % MiKTeX as graphics

% "Tikz is probably the most complex and powerful tool to create graphic elements."
% https://www.sharelatex.com/learn/TikZ_package
\usepackage{tikz} % MiKTeX as pgf

% packages used in the composition of the Scope of Approval part of the Valuation Model template
% multi-column typesetting mechanism
\usepackage{framed} % MiKTeX as framed
%\usepackage{mdframed} % MiKTeX as mdframed and l3kernel - the two are needed
%\def\changemargin#1#2{\list{}{\rightmargin#2\leftmargin#1}\item[]}
%\let\endchangemargin=\endlist
%\newmdenv[
  %skipabove=\parskip,
  %linewidth=0.8pt,
  %%innertopmargin = 0pt,
  %%leftmargin = 0pt,
  %%innerleftmargin = 1em,
  %%innerbottommargin = 0pt,
  %%innerrightmargin = 0pt,
  %%rightmargin = 0pt,
  %%topline = true,
  %%rightline = true,
  %%bottomline = true,
%]{simpleFrame}

% used in template development to fill in sections with random text
%\usepackage{lipsum} % MiKTeX as lipsum

% packages used to compose the header and footer of the MRG template
\usepackage{fancyhdr} % MiKTeX as fancyhdr

% to control itemized lists and enumerations
\usepackage{enumitem} %MiKTeX as enumitem

% standard array and tabular environments
\usepackage{array} % MiKTeX as tools

% packages to customize the tables of the MRG template
% colored table cells and headings
\usepackage{colortbl} % MiKTeX as colortbl
% tables that span over multiple pages
\usepackage{longtable} % MiKTeX as tools
% easier interface than tabular to control table size
\usepackage{tabularx} % MiKTeX as tools
% new column  definition to lighten the usage of tabularx
\newcolumntype{L}[1]{>{\hsize=#1\hsize\raggedright\arraybackslash}X}
\newcolumntype{R}[1]{>{\hsize=#1\hsize\raggedleft\arraybackslash}X}
\newcolumntype{C}[1]{>{\hsize=#1\hsize\centering\arraybackslash}X}
%\newcolumntype{C}{>{\centering\arraybackslash}X}
% writing professional tables
\usepackage{booktabs}

% The forloop package provides the command \forloop for the user.
\usepackage{forloop} % MiKTeX as forloop

\usepackage{xstring} % MiKTeX as xstring

\usepackage{etoolbox} % MiKTeX as etoolbox

\usepackage[ruled,vlined]{algorithm2e} % MiKTeX as algorithm2e

\usepackage{multirow} % MiKTeX as multirow

\makeatother

\begin{document}

\title{Notes on Risk Hedging}

\maketitle

\section{Dynamics Assumption}

Under the risk-neutral measure (Which one?), there exist two processes
modeling the underlying that are in practice both \textbf{\textcolor{red}{unknown}}
and \textbf{\textcolor{red}{different}}:
\begin{itemize}
\item True process: $S_{t}$ with the instantaneous volatility $\sigma_{t}$
that may be itself stochastic. This is the true process driving the
underlying
\item Market process: $S_{t}^{mkt}$. This is the hidden dynamics used by
the market to model the underlying.
\end{itemize}

\paragraph{\textcolor{red}{Important Note:}}

Here, the true dynamics does not refer to the dynamics of the underlying
under the real-world measure! It rather refers to the true behavior
of the instantaneous volatility of the underlying. 


\section{Delta-Hedging}

Considering a European option $V^{mkt}$ contingent on a stock $S$
with a payoff $f\left(S_{T}\right)$ at maturity $T$%
\footnote{The option price is labeled $V^{mkt}$ is labeled $mkt$ as it is
obtained using the market process while the underlying $S$ is driven
by the true process.%
}.

Considering a self-financing hedging portfolio comprised of the option,
the stock and cash. Its value at inception in given in the table $\ref{tab:deltaNeutralPort0}$.

\begin{table}[H]
\begin{centering}
\begin{tabular}{|c|c|}
\hline 
Components & Value\tabularnewline
\hline 
\hline 
Option & $V^{mkt}$\tabularnewline
\hline 
Stock & $-\Delta S$\tabularnewline
\hline 
Cash & -$V^{mkt}+\Delta S$\tabularnewline
\hline 
\end{tabular}
\par\end{centering}

\caption{Self-financing delta hedging portfolio at time $t=0$\label{tab:deltaNeutralPort0}}
\end{table}


Assuming excess cash is reinvested at the risk-free rate $r$ and
the stock $S$ is a giving a continuous dividend yield $q$, the value
of the portfolio at time $t=\delta t$ is given in the table $\ref{tab:deltaNeutralPort1}$.

\begin{table}[H]
\begin{centering}
\begin{tabular}{|c|c|}
\hline 
Components & Value\tabularnewline
\hline 
\hline 
Option & $V^{mkt}+\mathrm{d}V^{mkt}$\tabularnewline
\hline 
Stock & $-\Delta S-\Delta\delta S$\tabularnewline
\hline 
Cash & $\left(-V^{mkt}+\Delta S\right)\left(1+r\delta t\right)-q\delta t\Delta S$\tabularnewline
\hline 
\end{tabular}
\par\end{centering}

\caption{Self-financing delta hedging portfolio at time $t=\delta t$\label{tab:deltaNeutralPort1}}
\end{table}


Hence, the incremental delta-hedging P\&L is given by
\begin{eqnarray*}
\delta PNL & = & \mathrm{d}V^{mkt}-\Delta\delta S+\left(-V^{mkt}+\Delta S\right)r\delta t-q\delta t\Delta S
\end{eqnarray*}


Let us make some assumptions to further analyse the difference sources
of P\&L:
\begin{enumerate}
\item The Market Model is known.
\item The Market Model is a local volatility model.
\end{enumerate}
This first assumption implies that the perfect hedge ratio to cancel
first-order terms of $\delta S$ is known $\left(\frac{\partial V^{mkt}}{\partial S}\right)$.
The second assumption implies that the option price is simply a function
of the time and the underlying value $\left(V\left(t,S\right)\right)$.

Hence,
\[
\delta PNL=V^{mkt}\left(t+\delta t,S+\delta S\right)-V^{mkt}\left(t,S\right)-rV^{mkt}\delta t+\frac{\partial V^{mkt}}{\partial S}\left(-\delta S+rS\delta t-qS\delta t\right)
\]


The incremental $PNL$ is expanded to the lowest orders for $\delta t$
and $\delta S$ (this is equivalent to applying the Ito's lemma that
stops at second-order $\delta S^{2}$ terms)
\begin{eqnarray*}
\delta PNL & = & \frac{\partial V^{mkt}}{\partial t}\delta t+\frac{\partial V^{mkt}}{\partial S}\delta S+\frac{1}{2}\delta S^{2}\frac{\partial^{2}V^{mkt}}{\partial S^{2}}-rV^{mkt}\delta t+\frac{\partial V^{mkt}}{\partial S}\left(-\delta S+rS\delta t-qS\delta t\right)\\
 & = & \left(\frac{\partial V^{mkt}}{\partial t}-rV^{mkt}+\left(r-q\right)S\frac{\partial V^{mkt}}{\partial S}\right)\delta t+\frac{1}{2}\delta S^{2}\frac{\partial^{2}V^{mkt}}{\partial S^{2}}
\end{eqnarray*}


The second assumption also implies the existence of a volatility function
$\left(t,S\right)\rightarrow\hat{\sigma}\left(t,S\right)$ such that:

\begin{equation}
\frac{\partial V^{mkt}}{\partial t}\left(t,S\right)-rV^{mkt}\left(t,S\right)+\left(r-q\right)S\frac{\partial V^{mkt}}{\partial S}=-\frac{1}{2}\hat{\sigma}\left(t,S\right)^{2}S^{2}\frac{\partial^{2}V^{mkt}}{\partial S^{2}}\left(t,S\right)\label{eq:thetagamma}
\end{equation}


Hence, 

\begin{eqnarray*}
\delta PNL & = & \frac{1}{2}S^{2}\frac{\partial^{2}V^{mkt}}{\partial S^{2}}\left(t,S\right)\left(\frac{1}{S^{2}}\cdot\frac{\delta S^{2}}{\delta t}-\hat{\sigma}\left(t,S\right)^{2}\right)\delta t\\
 & = & \frac{1}{2}S^{2}\frac{\partial^{2}V^{mkt}}{\partial S^{2}}\left(t,S\right)\left(\sigma_{t}^{2}-\hat{\sigma}\left(t,S\right)^{2}\right)\delta t
\end{eqnarray*}


Even with the assumption of perfect knowledge of the market process,
the incremental delta-hedged P\&L is non null. In fact, we will still
be exposed to the discrepancy between the market belief of the true
volatility and the true volatility. 


\paragraph{Digression on the Calibration Problem}

Under the belief that the market is represented by an unique local
volatility function and that model is then used to price ALL securities
on the market%
\footnote{Here we reduce the market to a single underlying and we mean ALL securities
contingent to that underlying.%
}, that function can be extracted from the prices of vanilla option.
This is the so-called ``Dupire local volatility'' function.\\


This local volatility function that is obtained using the static picture
of a portion of the market should then always prevails. This implies
that applying the same calibration method at different times should
yield the same result. This is unfortunately not the case and the
assumption of a local volatility model collapses.\\


Hence, in a discrete delta-hedging setting, not only will we be exposed
to the error from assuming the delta to be piecewise constant, but
we will also be exposed to the calibration error that is making the
relation $\left(\ref{eq:thetagamma}\right)$ obsolete.


\paragraph{Digression on Model risk}

The P\&L approach is yet another angle from which one can observe
model risk, that is choosing a model is not able to capture risk factor
inherent to a derivative product.\\


In fact, assuming the market model indeed behaves as a Black-Scholes
model with a volatility parameter $\hat{\sigma}$, the hedging PNL
corresponding to a given option $V$ is:
\[
\delta PNL=\frac{1}{2}S^{2}\frac{\partial^{2}V^{mkt}}{\partial S^{2}}\left(t,S\right)\left(\frac{1}{S^{2}}\cdot\frac{\delta S^{2}}{\delta t}-\hat{\sigma}^{2}\right)\delta t
\]


It is not clear how that parameter can be estimated. In fact, for
a binary option, two constant volatility parameters may yield the
same market price but different incremental hedging PNL. It is clear
from the start that such assumption concerning the market dynamics
is wrong. 


\section{Gamma and Vega Hedging }

Now considering two exotic options $V^{mkt,1}$ and $V^{mkt,2}$ contingent
on the same underlying $S$ but with different payoff functions. The
objective is to build a portfolio comprised of the two options, the
underlying and cash that is delta and gamma-neutral. 

\begin{table}[H]
\begin{centering}
\begin{tabular}{|c|c|}
\hline 
Components & Value\tabularnewline
\hline 
\hline 
Option 1 & $V^{mkt,1}$\tabularnewline
\hline 
Option 2 & $-\lambda V^{mkt,2}$\tabularnewline
\hline 
Stock & $-\Delta S$\tabularnewline
\hline 
Cash & $-V^{mkt,1}+\lambda V^{mkt,2}+\Delta S$\tabularnewline
\hline 
\end{tabular}
\par\end{centering}

\caption{Self-financing delta and gamma hedging portfolio at time $t=0$}
\end{table}


\begin{table}[H]
\begin{centering}
\begin{tabular}{|c|c|}
\hline 
Components & Value\tabularnewline
\hline 
\hline 
Option 1 & $V^{mkt,1}+\mathrm{d}V^{mkt,1}$\tabularnewline
\hline 
Option 2 & $-\lambda V^{mkt,2}-\lambda\mathrm{d}V^{mkt,2}$\tabularnewline
\hline 
Stock & $-\Delta S-\Delta\delta S$\tabularnewline
\hline 
Cash & $\left(-V^{mkt,1}+\lambda V^{mkt,2}+\Delta S\right)\left(1+r\delta t\right)-q\delta t\Delta S$\tabularnewline
\hline 
\end{tabular}
\par\end{centering}

\caption{Self-financing delta and gamma hedging portfolio at time $t=\delta t$}
\end{table}


Hence, the incremental P\&L is given by:
\begin{eqnarray*}
\delta PNL & = & \mathrm{d}V^{mkt,1}-\lambda\mathrm{d}V^{mkt,2}-\Delta\delta S+\left(-V^{mkt,1}+\lambda V^{mkt,2}+\Delta S\right)r\delta t-q\delta t\Delta S\\
 & = & \mathrm{d}\left[V^{mkt,1}-\lambda V^{mkt,2}\right]-\left[V^{mkt,1}-\lambda V^{mkt,2}\right]r\delta t-\Delta\delta S+\left(r-q\right)\Delta S\delta t
\end{eqnarray*}


Given the discussion about the constant volatility or the local volatility
model, let us make new assumptions:
\begin{enumerate}
\item The Market Model is known
\item The Market Model is a two-factor model: one driven the stock while
the other drives the volatility. 
\end{enumerate}
Again the first assumption implies that the optimal hedge ratios are
known:
\begin{eqnarray*}
\Delta & = & \frac{\partial V^{mkt,1}}{\partial S}-\lambda\frac{\partial V^{mkt,2}}{\partial S}\\
\lambda & = & \frac{\frac{\partial^{2}V^{mkt,1}}{\partial S^{2}}}{\frac{\partial^{2}V^{mkt,2}}{\partial S^{2}}}
\end{eqnarray*}


The second assumption implies that the option values are now function
of 3 variables $V\left(t,S,\sigma\right)$
\begin{eqnarray*}
\delta PNL & = & \frac{\partial}{\partial t}\left[V^{mkt,1}-\lambda V^{mkt,2}\right]\delta t+\frac{\partial}{\partial S}\left[V^{mkt,1}-\lambda V^{mkt,2}\right]\delta S+\frac{1}{2}\delta S^{2}\frac{\partial^{2}}{\partial S^{2}}\left[V^{mkt,1}-\lambda V^{mkt,2}\right]\\
 &  & +\frac{\partial}{\partial\sigma}\left[V^{mkt,1}-\lambda V^{mkt,2}\right]\delta\sigma+\frac{1}{2}\delta\sigma^{2}\frac{\partial^{2}}{\partial\sigma^{2}}\left[V^{mkt,1}-\lambda V^{mkt,2}\right]+\frac{1}{2}\delta\sigma\delta S\frac{\partial^{2}}{\partial S\sigma}\left[V^{mkt,1}-\lambda V^{mkt,2}\right]\\
 &  & -\left[V^{mkt,1}-\lambda V^{mkt,2}\right]r\delta t-\Delta\delta S+\left(r-q\right)\Delta S\delta t\\
\delta PNL & = & \frac{\partial}{\partial t}\left[V^{mkt,1}-\lambda V^{mkt,2}\right]\delta t-\left[V^{mkt,1}-\lambda V^{mkt,2}\right]r\delta t+\left(r-q\right)\frac{\partial}{\partial S}\left[V^{mkt,1}-\lambda V^{mkt,2}\right]S\delta t\\
 &  & +\frac{\partial}{\partial\sigma}\left[V^{mkt,1}-\lambda V^{mkt,2}\right]\delta\sigma+\frac{1}{2}\delta\sigma^{2}\frac{\partial^{2}}{\partial\sigma^{2}}\left[V^{mkt,1}-\lambda V^{mkt,2}\right]+\frac{1}{2}\delta\sigma\delta S\frac{\partial^{2}}{\partial S\sigma}\left[V^{mkt,1}-\lambda V^{mkt,2}\right]
\end{eqnarray*}


We have (what is the interpretation? It feels like here we are dissociating
the dynamics of the implied volatility surface and its calibration.
Are they not linked?):
\begin{eqnarray*}
 &  & \frac{\partial}{\partial t}\left[V^{mkt,1}-\lambda V^{mkt,2}\right]\delta t-\left[V^{mkt,1}-\lambda V^{mkt,2}\right]r\delta t+\left(r-q\right)\frac{\partial}{\partial S}\left[V^{mkt,1}-\lambda V^{mkt,2}\right]S\delta t\\
 & = & -\frac{1}{2}\sigma^{2}S^{2}\frac{\partial^{2}}{\partial S^{2}}\left[V^{mkt,1}-\lambda V^{mkt,2}\right]\\
 & = & 0
\end{eqnarray*}


Hence,

\[
\delta PNL=\frac{\partial}{\partial\sigma}\left[V^{mkt,1}-\lambda V^{mkt,2}\right]\delta\sigma+\frac{1}{2}\delta\sigma^{2}\frac{\partial^{2}}{\partial\sigma^{2}}\left[V^{mkt,1}-\lambda V^{mkt,2}\right]+\frac{1}{2}\delta\sigma\delta S\frac{\partial^{2}}{\partial S\sigma}\left[V^{mkt,1}-\lambda V^{mkt,2}\right]
\]


``This is where stochastic volatility models are called for: their
aim is not to model the dynamics of realized volatility but to model
the dynamics of implied volatility.''
\end{document}
